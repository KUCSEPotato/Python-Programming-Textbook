\chapter{자료 구조}

지금까지 우리는
숫자, 문자열, 리스트와 같은 기본적인 자료형을 사용하여
데이터를 저장하고 처리해 왔습니다.
이러한 자료형들은
작은 프로그램을 작성하거나 간단한 문제를 해결하는 데에는 충분했습니다.

하지만 프로그램이 점점 커지고,
다뤄야 할 데이터의 양이 많아질수록
단순히 ``값을 저장하는 것''만으로는 부족해집니다.
이 시점부터는
\textbf{데이터를 어떻게 저장하느냐}가
\textbf{프로그램의 성능과 구조를 크게 좌우}하게 됩니다.

예를 들어,
\begin{itemize}
    \item 데이터를 차례대로 처리해야 하는 경우
    \item 먼저 들어온 데이터를 먼저 처리해야 하는 경우
    \item 계층적인 관계를 표현해야 하는 경우
    \item 여러 대상 사이의 복잡한 관계를 표현해야 하는 경우
\end{itemize}
와 같은 상황에서는,
리스트 하나만으로 문제를 해결하기 어렵습니다.

이러한 문제를 해결하기 위해 등장한 개념이
바로 \textbf{자료 구조(data structure)}입니다.

\begin{conceptbox}
자료 구조는
데이터를 단순히 ``저장''하는 것을 넘어,
\textbf{의미 있는 형태로 조직하고},
\textbf{효율적인 연산이 가능하도록 설계된 구조}입니다.
\end{conceptbox}

자료 구조를 선택하는 방식에 따라
\begin{itemize}
    \item 데이터를 얼마나 빠르게 찾을 수 있는지
    \item 삽입과 삭제가 얼마나 효율적인지
    \item 전체 프로그램의 시간 복잡도와 공간 복잡도가 어떻게 되는지
\end{itemize}
가 크게 달라집니다.

즉, 자료 구조는
\textbf{알고리즘의 성능을 결정하는 기반}이며,
좋은 프로그램을 만들기 위해 반드시 이해해야 할 핵심 개념입니다.

이 장에서는
단순한 자료형을 넘어,
\textbf{대표적인 자료 구조들의 동작 원리와 사용 목적}을 살펴봅니다.
구체적으로는 다음과 같은 자료 구조를 다룹니다.

\begin{itemize}
    \item \textbf{스택(Stack)}: 나중에 들어온 데이터가 먼저 처리되는 구조
    \item \textbf{큐(Queue)}: 먼저 들어온 데이터가 먼저 처리되는 구조
    \item \textbf{트리(Tree)}: 계층적인 관계를 표현하는 구조
    \item \textbf{그래프(Graph)}: 대상들 사이의 일반적인 관계를 표현하는 구조
\end{itemize}

이 장의 목표는
각 자료 구조의 정의를 암기하는 것이 아니라,
\begin{itemize}
    \item 언제 어떤 자료 구조를 사용해야 하는지 이해하고
    \item 각 구조가 어떤 문제에 적합한지 판단할 수 있으며
    \item 이후 배우게 될 알고리즘을 이해할 수 있는 기초를 다지는 것
\end{itemize}
입니다.

자료 구조를 이해하면,
단순히 ``작동하는 코드''를 넘어서
\textbf{효율적이고 구조적인 코드}를 작성할 수 있게 됩니다.