% =========================================================
\section{트리(Tree)}

트리(Tree)는
\textbf{계층적 구조(hierarchical structure)}를 표현하기 위한
대표적인 자료 구조입니다.

지금까지 배운
리스트, 스택, 큐는
데이터가 \textbf{일렬(linear)}로 나열된 구조였습니다.
즉, 각 데이터는
앞이나 뒤에 있는 데이터와만 직접적인 관계를 가집니다.

반면 트리는
데이터가
\textbf{부모-자식 관계(parent-child relationship)}로 연결되며,
전체 구조가
``가지가 뻗어 나가는 형태''를 가집니다.

\begin{conceptbox}
트리는 \textbf{노드(node)}와 노드 사이를 연결하는 \textbf{간선(edge)}으로 이루어진
\textbf{비선형(non-linear) 자료 구조}입니다.
하나의 노드는 여러 개의 자식을 가질 수 있으며, 이를 통해 계층 구조를 표현할 수 있습니다.
\end{conceptbox}

트리는
``상위 개념 아래에 하위 개념이 여러 개 존재하는 구조''
를 표현할 때 매우 자연스럽습니다.

% ---------------------------------------------------------
\subsection{트리의 기본 용어}

트리를 이해하기 위해서는
몇 가지 기본 용어를 먼저 익혀야 합니다.

\begin{itemize}
    \item \textbf{노드(node)} : 트리를 구성하는 각각의 데이터 단위

    \item \textbf{루트(root)} : 트리의 가장 위에 있는 시작 노드  
    하나의 트리에는 루트가 반드시 하나 존재합니다.

    \item \textbf{부모(parent)} / \textbf{자식(child)} :  
    직접 연결된 상위 노드를 부모, 하위 노드를 자식이라고 부릅니다.

    \item \textbf{형제(sibling)} : 같은 부모를 가지는 노드들

    \item \textbf{리프(leaf)} : 자식이 하나도 없는 노드  
    트리의 가장 끝에 위치합니다.

    \item \textbf{서브트리(subtree)} : 어떤 노드를 루트로 하는 작은 트리
\end{itemize}

\begin{notebox}
트리에서
\textbf{모든 노드는 루트로부터 정확히 하나의 경로만}을 가집니다.
이 점이 그래프와 트리를 구분하는 중요한 특징입니다.
\end{notebox}

% ---------------------------------------------------------
\subsection{이진 트리(Binary Tree)}

이진 트리는
트리 구조 중에서도
가장 기본적이고 중요한 형태입니다.

\begin{conceptbox}
\textbf{이진 트리(Binary Tree)}는
각 노드가
\textbf{최대 두 개의 자식 노드}만을 가질 수 있는 트리입니다.
이때 자식은
\textbf{왼쪽 자식(left child)}과
\textbf{오른쪽 자식(right child)}으로 구분됩니다.
\end{conceptbox}

중요한 점은,
\textbf{자식이 반드시 두 개일 필요는 없다}는 것입니다.

\begin{itemize}
    \item \textbf{자식 0개} : 리프 노드(leaf node)
    \item \textbf{자식 1개} : 한쪽으로만 연결된 노드
    \item \textbf{자식 2개} : 왼쪽과 오른쪽 자식을 모두 가진 노드
\end{itemize}

이진 트리는
다음과 같은 다양한 구조의 기반이 됩니다.

\begin{itemize}
    \item \textbf{이진 탐색 트리(Binary Search Tree, BST)} :  
    정렬된 데이터의 빠른 검색을 위해 사용

    \item \textbf{힙(Heap)} :  
    우선순위 큐(priority queue) 구현에 사용

    \item \textbf{표현식 트리(Expression Tree)} :  
    수식 계산 과정을 트리 형태로 표현
\end{itemize}

\begin{notebox}
이 장에서는
이진 트리의 \textbf{개념적 구조}까지만 다룹니다.
실제 구현과 탐색 알고리즘은
이후 장에서 자세히 설명합니다.
\end{notebox}

% ---------------------------------------------------------
\subsection{트리와 리스트의 차이}

트리는 리스트와 달리
\textbf{데이터의 위치가 인덱스로 정해지지 않습니다.}

\begin{itemize}
    \item \textbf{리스트} :  
    인덱스로 접근 (\texttt{list[0]}, \texttt{list[1]} 등)

    \item \textbf{트리} :  
    부모-자식 관계를 따라 구조적으로 이동
\end{itemize}

이 때문에 트리는
단순한 반복문보다는
\textbf{재귀(recursion)}나
\textbf{탐색 알고리즘}과 함께 자주 사용됩니다.

% ---------------------------------------------------------
\subsection{트리의 활용 예}

트리는 실제 시스템과 매우 밀접한 자료 구조입니다.

\begin{itemize}
    \item \textbf{파일 시스템} :  
    디렉터리와 파일의 계층 구조

    \item \textbf{HTML / XML 문서} :  
    태그가 중첩된 문서 구조

    \item \textbf{데이터베이스 인덱스} :  
    빠른 검색을 위한 트리 기반 구조 (B-Tree 등)

    \item \textbf{프로그래밍 언어 내부 구조} :  
    구문 분석 트리(AST, Abstract Syntax Tree)
\end{itemize}

\begin{notebox}
트리는
``정렬된 데이터 관리''와
``계층 구조 표현''
이라는 두 가지 관점에서
매우 중요한 역할을 합니다.
\end{notebox}

트리의 실제 구현과 순회 방법(전위/중위/후위 탐색)은  
클래스를 학습한 이후에 다시 다루겠습니다.