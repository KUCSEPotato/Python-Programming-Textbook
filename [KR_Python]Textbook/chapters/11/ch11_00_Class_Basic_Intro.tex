% 2/15 수정
\chapter{클래스 (Class) — 기초}

지금까지 우리는
정수, 문자열, 리스트, 딕셔너리와 같은
기본 자료형을 사용해 프로그램을 작성해 왔습니다.
또한 함수(function)를 이용하여 코드를 묶고, 재사용하고,
프로그램의 구조를 정리하는 방법도 배웠습니다.
하지만 프로그램이 점점 복잡해질수록,
단순히 ``데이터''와 ``함수''를 따로 관리하는 방식에는 한계가 생깁니다.

예를 들어,
어떤 학생을 표현하려 한다면 이름, 학번, 성적과 같은 \textit{데이터}가 필요하고, 그와 동시에 자기소개를 출력하는 기능, 평균 성적을 계산하는 기능과 같은 \textit{동작}도 필요합니다.
이처럼 \textbf{데이터와 그 데이터를 다루는 기능을 함께 묶어 관리할 필요}가 생기게 됩니다.

이 문제를 해결하는 개념이 바로
\textbf{클래스(Class)}입니다.

% ---------------------------------------------------------
\section*{객체(Object)란 무엇인가?}

현실 세계를 떠올려 봅시다.
우리는 사람, 자동차, 학생, 계좌와 같은
``대상''을 다룹니다.
이러한 대상은 두 가지 특징을 가집니다.

\begin{itemize}
  \item \textbf{상태(state)} — 그 대상이 가지고 있는 데이터
  \item \textbf{행동(behavior)} — 그 대상이 수행할 수 있는 기능
\end{itemize}

예를 들어 학생은

\begin{itemize}
  \item 이름과 학번이라는 상태를 가지고 있고
  \item 공부하거나 자기소개를 하는 행동을 수행할 수 있습니다.
\end{itemize}

이처럼
\textbf{상태와 행동을 하나의 단위로 묶은 것}을
프로그래밍에서는
\textbf{객체(Object)}라고 부릅니다.
객체는 단순한 값이 아니라,
\textbf{데이터와 기능이 결합된 존재}입니다.

% ---------------------------------------------------------
\section*{클래스(Class)란 무엇인가?}

\begin{conceptbox}
클래스(Class)는
객체를 만들기 위한 설계도(blueprint)입니다.
\end{conceptbox}

클래스는
객체가 어떤 데이터를 가지고,
어떤 동작을 할 수 있는지를 정의합니다.

클래스를 한 번 정의하면,
그 구조를 바탕으로
여러 개의 객체를 만들어낼 수 있습니다.

\begin{itemize}
  \item 클래스는 틀(template)
  \item 객체는 그 틀로부터 만들어진 실제 결과물
  \item 만들어진 객체를 \textbf{인스턴스(instance)}라고 부릅니다.
\end{itemize}

즉,
클래스는 ``개념''이고,
객체는 ``실제 존재''입니다.

% ---------------------------------------------------------
\section*{사실 우리는 이미 클래스를 사용하고 있다}

흥미로운 사실은,
우리가 지금까지 사용해 온
모든 기본 자료형도
사실은 클래스라는 점입니다.

\begin{itemize}
  \item \texttt{int}
  \item \texttt{str}
  \item \texttt{list}
  \item \texttt{dict}
\end{itemize}

예를 들어 다음 코드를 실행해 보면,

\begin{examplebox}
\begin{lstlisting}
s = "Hello"
print(type(s))
\end{lstlisting}

출력 결과(예시):
\begin{verbatim}
<class 'str'>
\end{verbatim}
\end{examplebox}

문자열 \texttt{"Hello"}는
\texttt{str} 클래스의 객체임을 알 수 있습니다.
즉, 우리는 이미 클래스의 인스턴스를 계속 사용해 왔던 것입니다.
이번 장에서는 이제 우리가 직접 클래스를 정의하고, 새로운 객체를 만들어 보게 됩니다.

% ---------------------------------------------------------
\section*{왜 클래스를 배워야 할까?}

클래스를 배우는 이유는 단순히 문법을 확장하기 위해서가 아닙니다.

클래스는

\begin{itemize}
  \item 프로그램을 더 구조적으로 만들고
  \item 관련 있는 데이터와 기능을 묶어 관리하게 하며
  \item 코드의 재사용성과 확장성을 높이고
  \item 복잡한 시스템을 단계적으로 설계할 수 있게 합니다.
\end{itemize}

이러한 설계 방식은 \textbf{객체 지향 프로그래밍(Object-Oriented Programming, OOP)}이라고 불립니다.
이번 장에서는 객체 지향의 모든 이론을 다루기보다는, \textbf{클래스를 직접 정의하고 사용하는 기초 문법}에 집중합니다.

% ---------------------------------------------------------
\section*{이번 장에서 다룰 내용}

이번 기초 장에서는 다음을 다룹니다.

\begin{itemize}
  \item 클래스의 기본 문법
  \item 생성자(\texttt{\_\_init\_\_})
  \item 인스턴스 변수
  \item 인스턴스 메서드
  \item 객체 생성과 사용
\end{itemize}

상속, 다형성, 클래스 변수, 특수 메서드 등은
다음 심화 장에서 다룰 예정입니다.

지금은
``클래스가 무엇인지''
그리고
``객체를 어떻게 만들고 사용하는지''
를 이해하는 데 집중해 봅시다.

\begin{notebox}
\textbf{객체 지향 프로그래밍(Object-Oriented Programming, OOP)}이란
프로그램을 ``객체''들의 상호작용으로 구성하는 설계 방식입니다.
각 객체는 자신의 \textbf{데이터(상태)}를 가지고 있으며, 그 데이터를 다루는 \textbf{기능(행동)}을 함께 포함합니다.

즉, \textbf{데이터와 기능을 분리하는 대신 하나의 단위로 묶어 설계하는 것}이 객체 지향 프로그래밍의 핵심 아이디어입니다.
이번 장에서는 이 객체 지향 설계의 출발점이 되는 ``클래스''를 배우게 됩니다.
\end{notebox}