% 2/15 수정 필요
\chapter{클래스 (Class) — 기초}

지금까지 우리는
정수, 문자열, 리스트, 딕셔너리와 같은
기본 자료형을 사용해 왔습니다.

또한 함수(function)를 이용하여
코드를 묶고, 재사용하고,
프로그램을 구조화하는 방법도 배웠습니다.

그러나 프로그램이 점점 복잡해지면
데이터와 그 데이터를 다루는 기능을
함께 묶어 관리할 필요가 생깁니다.

이때 사용하는 개념이 바로
\textbf{클래스(Class)}입니다.

% ---------------------------------------------------------
\section*{객체란 무엇인가?}

우리는 현실 세계에서
사람, 자동차, 학생, 계좌와 같은
``대상''을 다룹니다.

이러한 대상은

\begin{itemize}
  \item 상태(데이터)를 가지고 있고
  \item 행동(기능)을 수행할 수 있습니다.
\end{itemize}

예를 들어 학생은

\begin{itemize}
  \item 이름과 학번을 가지고 있으며
  \item 자기소개를 하거나 성적을 계산할 수 있습니다.
\end{itemize}

이처럼
\textbf{데이터와 기능을 하나로 묶은 단위}를
\textbf{객체(Object)}라고 합니다.

% ---------------------------------------------------------
\section*{클래스란 무엇인가?}

\begin{conceptbox}
클래스(Class)는
객체를 만들기 위한 설계도(blueprint)입니다.
\end{conceptbox}

클래스를 정의하면
그 구조를 기반으로
여러 개의 객체를 생성할 수 있습니다.

\begin{itemize}
  \item 클래스는 틀(template)
  \item 객체는 그 틀로부터 만들어진 실제 인스턴스(instance)
\end{itemize}

라고 이해하면 됩니다.

% ---------------------------------------------------------
\section*{사실 우리는 이미 클래스를 사용하고 있다}

지금까지 사용한 자료형들도
모두 클래스입니다.

\begin{itemize}
  \item \texttt{int}
  \item \texttt{str}
  \item \texttt{list}
  \item \texttt{dict}
\end{itemize}

예를 들어,

\begin{examplebox}
\begin{lstlisting}
s = "Hello"
type(s)
\end{lstlisting}
\end{examplebox}

의 결과는

\begin{verbatim}
<class 'str'>
\end{verbatim}

입니다.

즉, 문자열 역시
\texttt{str} 클래스의 객체인 것입니다.

이번 장에서는
이제 우리가 직접
클래스를 정의하고,
객체를 생성하는 방법을 배웁니다.

% ---------------------------------------------------------
\section*{이번 장에서 다룰 내용}

이번 기초 장에서는 다음을 다룹니다.

\begin{itemize}
  \item 클래스의 기본 문법
  \item 생성자(\texttt{\_\_init\_\_})
  \item 인스턴스 변수
  \item 인스턴스 메서드
  \item 객체 생성과 사용
\end{itemize}

상속, 다형성, 고급 메서드 설계 등은
다음 심화 장에서 다룰 예정입니다.

우선은
``클래스가 무엇인지''
그리고
``객체를 어떻게 만들고 사용하는지''
에 집중해 봅시다.