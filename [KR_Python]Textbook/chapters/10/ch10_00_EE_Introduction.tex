% =========================================================
\chapter{에러(Error)와 예외(Exception)}

프로그램을 작성하다 보면
항상 원하는 대로만 동작하지는 않습니다.
문법을 잘못 작성했을 수도 있고,
존재하지 않는 파일을 열려고 할 수도 있으며,
0으로 나누는 연산을 수행할 수도 있습니다.
이처럼 프로그램 실행 과정에서 발생하는 문제를
통틀어 \textbf{에러(Error)}라고 부릅니다.
그러나 모든 에러가 동일한 의미를 가지는 것은 아닙니다.

\begin{conceptbox}
파이썬에서 발생하는 문제는 크게 두 가지로 나눌 수 있습니다.

\begin{itemize}
  \item \textbf{문법 오류(Syntax Error)} :
        코드 자체가 잘못되어 실행조차 되지 않는 경우
  \item \textbf{예외(Exception)} :
        문법은 올바르지만 실행 도중 문제가 발생하는 경우
\end{itemize}
\end{conceptbox}

예외는 프로그램 실행 중 언제든지 발생할 수 있습니다.
중요한 점은,
\textbf{예외는 통제 불가능한 사고가 아니라 처리할 수 있는 상황이라는 것}입니다.

이번 장에서는

\begin{itemize}
  \item 에러와 예외의 차이
  \item 자주 등장하는 예외 유형
  \item \texttt{try}–\texttt{except} 문을 이용한 예외 처리
  \item \texttt{finally}, \texttt{else}의 동작 방식
  \item 사용자 정의 예외(Custom Exception)
\end{itemize}

를 차례대로 살펴보겠습니다.

이 장을 마치면,
프로그램이 실패했을 때 당황하는 대신
그 상황을 \textbf{설계의 일부로 다룰 수 있게 될 것}입니다.