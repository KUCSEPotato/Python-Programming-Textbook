% =========================================================
\section{예외(Exception)}

프로그램을 작성하다 보면
문법은 올바르지만 실행 도중 문제가 발생하는 경우가 있습니다.
이처럼 실행 중 발생하는 오류를
\textbf{예외(Exception)}라고 합니다.

문법 오류(Syntax Error)는
코드를 수정하지 않으면 실행 자체가 불가능하지만,

예외는 프로그램 내부에서
\textbf{직접 처리(handle)}할 수 있습니다.

예외 처리를 통해
프로그램이 갑자기 종료되는 것을 막고,
보다 안정적인 동작을 구현할 수 있습니다.

% ---------------------------------------------------------
\subsection{예외가 발생하면 어떻게 될까요?}

예외를 처리하지 않으면
프로그램은 즉시 중단됩니다.

\begin{examplebox}
\begin{lstlisting}
print("Program starts")
print(10 / 0)
print("Program ends")
\end{lstlisting}

출력 결과(예시):
\begin{verbatim}
Program starts
Traceback (most recent call last):
  File "/Users/potato/.../test.py", line 2, in <module>
    print(10 / 0)
          ~~~^~~
ZeroDivisionError: division by zero
\end{verbatim}
\end{examplebox}

위 코드에서는
\texttt{ZeroDivisionError}가 발생하며
마지막 줄은 실행되지 않습니다.

이처럼 예외가 발생하면
현재 실행 흐름은 즉시 멈추고,
에러 메시지를 출력한 뒤 프로그램이 종료됩니다.
하지만 항상 이렇게 종료되는 것이 바람직한 것은 아닙니다.
사용자로부터 잘못된 입력을 받았을 때,
프로그램이 완전히 종료되기보다는
다시 입력을 받는 편이 더 적절할 수 있습니다.
이를 위해 사용하는 구조가
\textbf{try-except 문}입니다.

% ---------------------------------------------------------
\subsection{try-except 기본 구조}

try-except 구문의 기본 구조는 다음과 같습니다.

\begin{examplebox}
\begin{lstlisting}
try:
    statement
except ExceptionType:
    handler
\end{lstlisting}
\end{examplebox}

\texttt{try} 블록 안에는
예외가 발생할 가능성이 있는 코드를 작성합니다.
예외가 발생하면,
해당 예외 타입에 맞는 \texttt{except} 블록이 실행됩니다.

\begin{examplebox}
\begin{lstlisting}
try:
    result = 10 / 0
except ZeroDivisionError:
    print("Cannot divide by zero.")
\end{lstlisting}

출력 결과(예시):
\begin{verbatim}
Cannot divide by zero.
\end{verbatim}
\end{examplebox}

이제 프로그램은 중단되지 않고,
예외를 감지하여 지정한 코드로 대응합니다.

% ---------------------------------------------------------
\subsection{여러 예외 처리하기}

하나의 \texttt{try} 블록에는 여러 개의 \texttt{except} 블록을 둘 수 있습니다.
예를 들어, 잘못된 형 변환에서 발생하는 예외와 0으로 나누었을 때 발생하는 예외를 각각 구분하여 처리해 보겠습니다.

\begin{examplebox}
\begin{lstlisting}
try:
    number = int(input("Enter a number: "))
    print(10 / number)
except ValueError:
    print("You must enter an integer.")
except ZeroDivisionError:
    print("Cannot divide by zero.")
\end{lstlisting}
\end{examplebox}

이처럼 각 예외 타입에 따라 다른 방식으로 대응할 수 있습니다. 
예외 타입을 구체적으로 명시하는 것이 좋은 습관입니다.

\begin{notebox}
위 예시 코드에서는 숫자로 변환할 수 없는 값이 입력되면 \texttt{ValueError}가 발생하고,
0이 입력되어 나눗셈을 수행할 경우 \texttt{ZeroDivisionError}가 발생합니다.
\end{notebox}

% ---------------------------------------------------------
\subsection{예외 객체 받기}

\texttt{except} 블록에서는
발생한 예외 객체를 변수로 받아
자세한 정보를 확인할 수 있습니다.

기본 구조는 다음과 같습니다.

\begin{center}
\texttt{except 예외타입 as 변수이름:}
\end{center}

이때 변수에는
발생한 예외 객체가 그대로 전달됩니다.

\begin{examplebox}
\begin{lstlisting}
try:
    value = int("abc")
except ValueError as error:
    print("Error message:", error)
\end{lstlisting}

출력 결과(예시):
\begin{verbatim}
Error message: invalid literal for int() with base 10: 'abc'
\end{verbatim}
\end{examplebox}

\texttt{error} 변수에는
예외 객체가 담기며,
이를 출력하면 예외의 상세 메시지를 확인할 수 있습니다.

\begin{conceptbox}
예외도 하나의 객체(object)입니다.
따라서 변수에 저장하고,
출력하거나,
필요한 속성을 확인할 수 있습니다.
\end{conceptbox}

이 기능은 다음과 같은 상황에서 특히 유용합니다.

\begin{itemize}
  \item 사용자에게 구체적인 오류 메시지를 보여주고 싶을 때
  \item 로그(log)에 상세한 오류 정보를 남기고 싶을 때
  \item 예외 메시지를 분석하여 추가 처리를 하고 싶을 때
\end{itemize}

\begin{notebox}
초보 단계에서는
예외 타입만 처리해도 충분하지만,
실제 프로그램에서는
예외 객체의 메시지를 함께 활용하는 습관이 중요합니다.
\end{notebox}

% ---------------------------------------------------------
\subsection{else와 finally}

\texttt{try-except} 문에는
선택적으로 \texttt{else}와 \texttt{finally} 블록을 함께 사용할 수 있습니다.

\begin{conceptbox}
\begin{itemize}
  \item \texttt{else} :
        \textbf{예외가 발생하지 않았을 때만} 실행
  \item \texttt{finally} :
        \textbf{예외 발생 여부와 관계없이 항상} 실행
\end{itemize}
\end{conceptbox}

전체 구조는 다음과 같습니다.

\begin{center}
\texttt{try → except → else → finally}
\end{center}

각 블록의 실행 조건은 다음과 같습니다.

\begin{itemize}
  \item 예외 발생 → \texttt{except} 실행 → \texttt{finally} 실행
  \item 예외 없음 → \texttt{else} 실행 → \texttt{finally} 실행
\end{itemize}

\begin{examplebox}
\begin{lstlisting}
try:
    number = int("10")
except ValueError:
    print("Conversion failed.")
else:
    print("Conversion succeeded.")
finally:
    print("This block always runs.")
\end{lstlisting}

출력 결과(예시):
\begin{verbatim}
Conversion succeeded.
This block always runs.
\end{verbatim}
\end{examplebox}

\texttt{else} 블록은
``정상적으로 실행되었을 때만'' 수행하고 싶은 코드를
분리할 때 유용합니다.

\texttt{finally} 블록은
파일 정리, 네트워크 연결 종료, 자원 해제 등
\textbf{반드시 실행되어야 하는 코드}를 작성할 때 사용됩니다.

\begin{notebox}
\texttt{finally}는 예외가 발생하더라도 실행됩니다.
심지어 \texttt{return} 문이 있거나,
프로그램이 종료되는 상황에서도
가능한 한 실행을 보장합니다.
\end{notebox}

\begin{notebox}
파일 입출력에서 사용했던 \texttt{with} 문 역시
내부적으로는 이러한 ``정리 보장'' 개념을 활용합니다.
즉, \texttt{with}는 자원 해제를 자동으로 처리해 주는 구조입니다.
\end{notebox}

% ---------------------------------------------------------
\subsection{모든 예외 잡기 (주의)}

예외의 구체적인 종류를 지정하지 않고,
\texttt{Exception}을 사용하면
대부분의 일반적인 예외를 한 번에 처리할 수 있습니다.

\begin{examplebox}
\begin{lstlisting}
try:
    risky_operation()
except Exception:
    print("An error occurred.")
\end{lstlisting}
\end{examplebox}

\texttt{Exception}은
대부분의 표준 예외들의 상위 클래스입니다.
따라서 위와 같이 작성하면
\textbf{거의 모든 일반적인 예외를 잡을 수 있습니다.}
하지만 이 방식은 매우 신중하게 사용해야 합니다.

모든 예외를 한꺼번에 잡아버리면

\begin{itemize}
  \item 실제로 어떤 오류가 발생했는지 알 수 없고
  \item 버그를 조기에 발견하지 못할 수 있으며
  \item 프로그램이 잘못된 상태로 계속 실행될 위험이 있습니다.
\end{itemize}

그러므로, 가능한 한 \textbf{구체적인 예외 타입을 명시하는 습관}을 들이세요.

예를 들어 다음과 같이 구체적으로 작성하는 편이 좋습니다.

\begin{examplebox}
\begin{lstlisting}
try:
    number = int(user_input)
    result = 10 / number
except ValueError:
    print("Invalid number.")
except ZeroDivisionError:
    print("Cannot divide by zero.")
\end{lstlisting}
\end{examplebox}

\begin{notebox}
예외를 넓게 잡아야 하는 경우도 있습니다.
예를 들어 최상위(main) 레벨에서
프로그램이 갑자기 종료되지 않도록
마지막 안전 장치로 사용하는 경우입니다.
그러나 이 경우에도
예외 정보를 로그로 기록하는 것이 바람직합니다.
\end{notebox}

\begin{exercisebox}
다음 코드에서 발생할 수 있는 예외를 생각해 보고, 각각에 맞는 \texttt{except} 블록을 작성해 보세요.

\begin{lstlisting}
data = ["10", "20", "abc"]
for item in data:
    number = int(item)
    print(100 / number)
\end{lstlisting}
\end{exercisebox}

\begin{exercisebox}
\texttt{try-except-else-finally} 구조를 사용하여 다음 조건을 만족하는 코드를 작성하세요.

\begin{itemize}
  \item 정수 변환이 성공하면 "Success" 출력
  \item 실패하면 "Failure" 출력
  \item 마지막에는 항상 "Done" 출력
\end{itemize}
\end{exercisebox}

% ---------------------------------------------------------
\begin{readernotebox}
예외 단원의 핵심을 스스로 점검해 보세요.

\begin{itemize}
  \item 에러(Error)와 예외(Exception)의 차이는 무엇인가요?

  \vspace{0.6cm}

  \item 왜 가능한 한 구체적인 예외 타입을 명시하는 것이 좋을까요?

  \vspace{0.6cm}

  \item \texttt{else} 블록은 언제 실행되며,
        \texttt{finally} 블록과 어떤 차이가 있나요?

  \vspace{0.6cm}

  \item 모든 예외를 한꺼번에 잡는 것이 위험한 이유는 무엇인가요?
\end{itemize}

\end{readernotebox}