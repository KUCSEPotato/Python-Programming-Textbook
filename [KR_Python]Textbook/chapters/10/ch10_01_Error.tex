% ---------------------------------------------------------
\section{에러(Error)}

프로그램은 사람이 작성합니다.
따라서 실수가 발생하는 것은 자연스러운 일입니다.

파이썬에서 프로그램 실행이 정상적으로 이루어지지 않을 때,
이를 통틀어 \textbf{에러(Error)}라고 부릅니다.

에러는 크게 두 가지로 나눌 수 있습니다.

\begin{itemize}
  \item \textbf{문법 오류(Syntax Error)}
  \item \textbf{실행 중 오류(Runtime Error)}
\end{itemize}

이 두 가지는 발생 시점과 성격이 다릅니다.

% ---------------------------------------------------------
\subsection{문법 오류 (Syntax Error)}

문법 오류는
파이썬 문법 규칙에 어긋난 코드를 작성했을 때 발생합니다.

이 경우,
코드는 실행되기도 전에 인터프리터에 의해 중단됩니다.

\begin{examplebox}
\begin{lstlisting}
if True
    print("Hello")
\end{lstlisting}
\end{examplebox}

위 코드는 콜론(:)이 빠졌기 때문에
다음과 같은 에러가 발생합니다.

\begin{examplebox}
출력 결과(예시):
\begin{verbatim}
Traceback (most recent call last):
  File "/Users/potato/.../Python-Programming-Textbook/test.py", line 1
    if True
           ^
SyntaxError: expected ':'
\end{verbatim}
\end{examplebox}

\begin{notebox}
문법 오류는 프로그램이 시작조차 되지 않기 때문에
\textbf{예외 처리로 잡을 수 없습니다.}
코드를 수정해야만 해결됩니다.
\end{notebox}

% ---------------------------------------------------------
\subsection{실행 중 오류 (Runtime Error)}

문법은 올바르지만
프로그램 실행 도중 문제가 발생하는 경우를
\textbf{실행 중 오류(Runtime Error)}라고 합니다.

이때 발생하는 오류를
파이썬에서는 \textbf{예외(Exception)}라고 부릅니다.

\begin{examplebox}
\begin{lstlisting}
print(10 / 0)
\end{lstlisting}

출력 결과(예시):

\begin{verbatim}
Traceback (most recent call last):
  File "/Users/potato/.../test.py", line 1, in <module>
    print(10 / 0)
          ~~~^~~
ZeroDivisionError: division by zero
\end{verbatim}
\end{examplebox}

이 코드는 문법적으로는 올바르지만,
수학적으로 허용되지 않는 연산을 수행했기 때문에
실행 도중 오류가 발생합니다.

% ---------------------------------------------------------
\subsection{Traceback 읽기}

실행 중 오류가 발생하면,
파이썬은 오류가 발생한 위치를 함께 출력합니다.
이를 \textbf{Traceback}이라고 합니다.

Traceback은
\begin{itemize}
  \item 어느 파일에서
  \item 몇 번째 줄에서
  \item 어떤 오류가 발생했는지
\end{itemize}
를 알려주는 정보입니다.

\begin{notebox}
에러 메시지를 두려워하지 마십시오.
Traceback은 문제를 해결하기 위한
\textbf{가장 중요한 단서}입니다.
\end{notebox}