% ---------------------------------------------------------
\section{패키지 관리 도구: \texttt{pip}}

지금까지는
패키지의 구조와 import 방식에 대해 살펴보았습니다.
하지만 실제 개발에서는
필요한 패키지를 직접 작성하기보다,
이미 공개된 패키지를 설치하여 사용하는 경우가 많습니다.

이때 사용되는 도구가
\textbf{패키지 관리 도구(package manager)}이며,
파이썬에서는 \texttt{pip}가 표준적으로 사용됩니다.

\begin{conceptbox}
\texttt{pip}는
외부 패키지를 다운로드하고 설치하며,
필요한 버전을 관리해 주는 도구입니다.
설치된 패키지는 보통
\texttt{site-packages} 디렉터리에 저장됩니다.
\end{conceptbox}

% ---------------------------------------------------------
\subsection{패키지 설치}

터미널에서 다음 명령을 실행하면,
\texttt{requests} 패키지가 설치됩니다.

\begin{examplebox}
\begin{lstlisting}
pip install requests
\end{lstlisting}
\end{examplebox}

설치가 완료되면,
해당 패키지는
다른 모듈과 동일한 방식으로 import할 수 있습니다.

\begin{examplebox}
\begin{lstlisting}
import requests

response = requests.get("https://example.com")
print(response.status_code)
\end{lstlisting}

출력 결과(예시):
\begin{verbatim}
200
\end{verbatim}
\end{examplebox}

\begin{notebox}
\texttt{pip}로 설치한 패키지는
현재 사용 중인 파이썬 환경에만 영향을 미칩니다.
따라서 어떤 환경에서 설치했는지가 매우 중요합니다.
\end{notebox}

% ---------------------------------------------------------
\subsection{설치된 패키지 확인}

현재 환경에 설치된 패키지 목록은
다음 명령으로 확인할 수 있습니다.

\begin{examplebox}
\begin{lstlisting}
pip list
\end{lstlisting}
\end{examplebox}

또한,
특정 패키지의 상세 정보를 확인하려면
다음과 같이 사용할 수 있습니다.

\begin{examplebox}
\begin{lstlisting}
pip show requests
\end{lstlisting}
\end{examplebox}

% ---------------------------------------------------------
\section{가상 환경(Virtual Environment, \texttt{venv})}

여러 프로젝트를 진행하다 보면,
서로 다른 프로젝트가
\textbf{서로 다른 패키지 버전}을 요구하는 경우가 자주 발생합니다.
이를 하나의 파이썬 환경에서 관리하면
패키지 충돌 문제가 발생할 수 있습니다.

이 문제를 해결하기 위해
\textbf{가상 환경(virtual environment)}을 사용합니다.

\begin{conceptbox}
가상 환경은
프로젝트별로 독립된 파이썬 실행 환경을 만들어,
패키지 설치와 버전을 서로 분리해 주는 기능입니다.
\end{conceptbox}

% ---------------------------------------------------------
\subsection{가상 환경 생성}

다음 명령은
현재 디렉터리에 \texttt{venv}라는 이름의
가상 환경을 생성합니다.

\begin{examplebox}
\begin{lstlisting}
python3 -m venv venv
\end{lstlisting}
\end{examplebox}

이 명령을 실행하면,
파이썬 실행 파일과
독립적인 \texttt{site-packages} 디렉터리가
함께 생성됩니다.

% ---------------------------------------------------------
\subsection{가상 환경 활성화}

가상 환경을 사용하려면,
먼저 활성화해야 합니다.

\subsection*{macOS / Linux}

\begin{examplebox}
\begin{lstlisting}
source venv/bin/activate
\end{lstlisting}
\end{examplebox}

\subsection*{Windows}

\begin{examplebox}
\begin{lstlisting}
venv\Scripts\activate
\end{lstlisting}
\end{examplebox}

가상 환경이 활성화되면,
터미널 프롬프트 앞에
환경 이름이 표시됩니다.

\begin{examplebox}
\begin{verbatim}
(venv) potato@Potato-MacBookAir [KR_Python]Textbook % 
\end{verbatim}
\end{examplebox}

가상 환경을 삭제하려면,
터미널에서 다음 명령을 실행합니다.

\begin{examplebox}
\begin{lstlisting}
rm -rf venv
\end{lstlisting}
\end{examplebox}

\begin{notebox}
\texttt{rm -rf} 명령은
지정한 디렉터리를
확인 없이 강제로 삭제합니다.
따라서 현재 위치와
삭제 대상 디렉터리 이름을
반드시 다시 한 번 확인한 후 실행해야 합니다.
\end{notebox}

% ---------------------------------------------------------
\subsection{가상 환경에서의 패키지 설치}

가상 환경이 활성화된 상태에서
\texttt{pip install}을 실행하면,
패키지는 해당 가상 환경에만 설치됩니다.

\begin{examplebox}
\begin{lstlisting}
pip install requests
\end{lstlisting}
\end{examplebox}

이렇게 하면,
다른 프로젝트나
전역 파이썬 환경에는
아무런 영향을 주지 않습니다.

\begin{notebox}
실제 프로젝트에서는
항상 가상 환경을 생성한 뒤
그 안에서 패키지를 설치하는 것이
권장되는 개발 습관입니다.
\end{notebox}

\begin{readernotebox}
패키지와 가상 환경의 핵심 개념을 스스로 점검해 보세요.
\begin{itemize}
  \item 파이썬에서
        패키지(package)와
        모듈(module)은
        어떤 점에서 다르며,
        왜 패키지 구조가 필요한가요?
  \vspace{0.6cm}

  \item 표준 라이브러리와
        서드파티 패키지는
        어떤 기준으로 구분되나요?
        각각은 언제 사용하는 것이 적절한가요?
  \vspace{0.6cm}

  \item \texttt{pip}는
        어떤 역할을 하는 도구인가요?
        단순히 \texttt{import}와는
        어떤 점에서 다른가요?
  \vspace{0.6cm}

  \item 가상 환경(\texttt{venv})을
        사용하지 않고
        여러 프로젝트를 관리하면
        어떤 문제가 발생할 수 있을까요?
  \vspace{0.6cm}

  \item 가상 환경이 활성화된 상태에서
        \texttt{pip install}을 실행했을 때,
        패키지는
        어디에 설치되며,
        다른 프로젝트에는
        어떤 영향을 주나요?
\end{itemize}
\end{readernotebox}