% ---------------------------------------------------------
\section{패키지(Package)}

프로그램의 규모가 커지면,
하나의 파일이나 몇 개의 모듈만으로는
코드를 체계적으로 관리하기 어려워집니다.
이때 사용하는 구조가
\textbf{패키지(package)}입니다.

\begin{conceptbox}
\textbf{패키지(package)}는
\textbf{여러 모듈을 디렉터리 단위로 묶어}
이름 공간(namespace)을 계층적으로 구성하는 방법입니다.
패키지와 모듈은
\textbf{점(\texttt{.})으로 구분된 이름}으로 접근합니다.
\end{conceptbox}

예를 들어,
\texttt{A.B}라는 이름은
\texttt{A}라는 패키지 안에 있는
\texttt{B}라는 모듈(또는 서브 패키지)을 의미합니다.

이러한 구조 덕분에,
여러 개발자가 만든 모듈들이
서로의 이름 충돌을 걱정하지 않고
하나의 프로젝트 안에서 공존할 수 있습니다.
실제로 \texttt{NumPy}, \texttt{Pillow}와 같은 대형 라이브러리들은
모두 패키지 구조로 구성되어 있습니다.

% ---------------------------------------------------------
\subsection{패키지의 디렉터리 구조}

패키지는
파일 시스템 상에서
\textbf{디렉터리 구조}로 표현됩니다.
다음은 음향 파일을 처리하는 패키지를
예시로 한 구조입니다.

\begin{examplebox}
\begin{verbatim}
sound/
    __init__.py
    formats/
        __init__.py
        wavread.py
        wavwrite.py
        aiffread.py
        aiffwrite.py
    effects/
        __init__.py
        echo.py
        surround.py
        reverse.py
    filters/
        __init__.py
        equalizer.py
        vocoder.py
        karaoke.py
\end{verbatim}
\end{examplebox}

이 구조에서:

\begin{itemize}
  \item \texttt{sound}는 최상위 패키지
  \item \texttt{formats}, \texttt{effects}, \texttt{filters}는 서브 패키지
  \item 각 \texttt{.py} 파일은 하나의 모듈
\end{itemize}

입니다.

% ---------------------------------------------------------
\subsection{\texttt{\_\_init\_\_.py}의 역할}

파이썬이
어떤 디렉터리를
\textbf{패키지로 인식하도록 만들기 위해}
전통적으로 사용되는 파일이
\texttt{\_\_init\_\_.py}입니다.

\begin{conceptbox}
\texttt{\_\_init\_\_.py} 파일이 존재하면,
해당 디렉터리는
파이썬에 의해 패키지로 취급됩니다.
\end{conceptbox}

가장 단순한 경우,
\texttt{\_\_init\_\_.py}는
빈 파일이어도 무방합니다.
하지만 이 파일 안에서:

\begin{itemize}
  \item 패키지 초기화 코드 실행
  \item 패키지에서 공개할 이름 제어
\end{itemize}

와 같은 작업을 수행할 수도 있습니다.

% ---------------------------------------------------------
\subsection{패키지 임포트 방법}

패키지에 포함된 모듈은
여러 가지 방식으로 import할 수 있습니다.

\subsubsection{전체 경로를 사용하는 방식}

\begin{examplebox}
\begin{lstlisting}
import sound.effects.echo

sound.effects.echo.echofilter(input, output)
\end{lstlisting}
\end{examplebox}

이 방식은
모듈의 전체 경로를 명시하므로
출처가 가장 분명합니다.

\subsubsection{서브 모듈만 가져오는 방식}

\begin{examplebox}
\begin{lstlisting}
from sound.effects import echo

echo.echofilter(input, output)
\end{lstlisting}
\end{examplebox}

이 경우,
\texttt{echo} 모듈만 현재 이름 공간으로 가져옵니다.

\subsubsection{특정 이름만 가져오는 방식}

\begin{examplebox}
\begin{lstlisting}
from sound.effects.echo import echofilter

echofilter(input, output)
\end{lstlisting}
\end{examplebox}

이 방식은
함수나 변수 하나만 직접 사용할 수 있어
가장 간결하지만,
출처가 코드에서 드러나지 않을 수 있습니다.

\begin{notebox}
\texttt{from package import item}에서
\texttt{item}은
서브 모듈, 서브 패키지,
또는 패키지 내부에서 정의된 다른 이름일 수 있습니다.
\end{notebox}

% ---------------------------------------------------------
\subsection{\texttt{from package import *}와 \texttt{\_\_all\_\_}}

패키지에서도
\texttt{from package import *} 문법을 사용할 수 있습니다.
하지만 이 동작은
\texttt{\_\_init\_\_.py}에 정의된
\texttt{\_\_all\_\_} 변수에 의해 제어됩니다.

\begin{examplebox}
\begin{lstlisting}
# sound/effects/__init__.py
__all__ = ["echo", "surround", "reverse"]
\end{lstlisting}
\end{examplebox}

이렇게 정의되어 있으면,

\begin{examplebox}
\begin{lstlisting}
from sound.effects import *
\end{lstlisting}
\end{examplebox}

는
\texttt{echo}, \texttt{surround}, \texttt{reverse} 모듈만을
import하게 됩니다.

\begin{notebox}
\texttt{\_\_all\_\_}이 정의되어 있지 않다면,
\texttt{import *}는
패키지에 포함된 모든 서브 모듈을 자동으로 불러오지 않습니다.
일반적으로 이 방식은
가독성과 안정성 측면에서 권장되지 않습니다.
\end{notebox}

% ---------------------------------------------------------
\subsection{패키지 내부 간의 임포트}

패키지 내부의 모듈들은
서로를 import할 수 있습니다.
이때 두 가지 방식이 존재합니다.

\subsubsection{절대 임포트(Absolute Import)}

\begin{examplebox}
\begin{lstlisting}
from sound.effects import echo
\end{lstlisting}
\end{examplebox}

절대 임포트는
항상 패키지의 최상위 이름부터 시작하며,
가장 명확하고 권장되는 방식입니다.

\subsubsection{상대 임포트(Relative Import)}

\begin{examplebox}
\begin{lstlisting}
from . import echo
from ..filters import equalizer
\end{lstlisting}
\end{examplebox}

상대 임포트는
현재 모듈을 기준으로
점(\texttt{.})을 사용해
패키지 내부 위치를 나타냅니다.

\begin{notebox}
상대 임포트는
모듈이 패키지의 일부로 import될 때만 동작합니다.
따라서 프로그램의 진입점이 되는 모듈에서는
절대 임포트를 사용하는 것이 안전합니다.
\end{notebox}