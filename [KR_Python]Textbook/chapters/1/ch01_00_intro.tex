\chapter{Introduction to Python}
\textbf{Python의 세계에 오신 여러분들을 환영합니다!} \newline
새로운 언어를 배우는 일은 언제나 설레는 경험입니다. 
프로그래밍을 처음 접하는 분들 중에는 이를 어렵게 느끼는 경우도 많지만,
프로그래밍은 본질적으로 컴퓨터와 소통하는 과정이라고 생각할 수 있습니다.

사람과 사람이 서로 대화하기 위해 언어가 필요하듯이,
컴퓨터와 대화하기 위해서도 프로그래밍 언어가 필요합니다.
이번 교과서에서 다룰 Python은 문법이 간결하고 읽기 쉬워,
프로그래밍을 처음 배우는 분들에게 특히 적합한 언어입니다.

이제 저와 함께 Python의 세계로 한 걸음씩 들어가 봅시다.

\section{Python이란?}

\begin{conceptbox}
Python은 고수준(high-level), 인터프리티드(interpreted) 프로그래밍 언어로, 사람이 읽고 이해하기 쉬운 문법과 높은 생산성을 목표로 설계되었습니다.
\end{conceptbox}

Python은 현대에 가장 널리 사용되는 프로그래밍 언어 중 하나로, 프로그래밍을 처음 접하려는 사람부터 전문 개발자에 이르기까지 폭넓은 사용자층을 보유하고 있습니다.
문법이 직관적이고 코드의 가독성이 뛰어나기 때문에, 프로그래밍 언어를 처음 배우는 입문자에게 특히 적합합니다.

또한 Python은 다양한 표준 라이브러리와 방대한 외부 패키지 생태계를 바탕으로 웹 개발, 데이터 분석, 인공지능, 자동화 스크립트, 과학 계산 등 여러 분야에서 폭넓게 활용되고 있습니다.
이러한 특성 덕분에 Python은 단순한 학습용 언어를 넘어, 실무와 연구 현장에서도 핵심 도구로 자리 잡았습니다.

물론 Python에도 한계는 존재합니다.
일반적으로 Python은 C 언어와 같은 컴파일 언어에 비해 실행 속도가 느린 편이며, 상황에 따라 수십 배 이상의 성능 차이가 발생할 수 있습니다.
이는 대부분의 프로그래밍 언어에서 \emph{사용의 편의성}과 \emph{실행 성능}이 서로 트레이드오프 관계에 있기 때문입니다.
이러한 이유로 대규모 게임 엔진이나 실시간 성능이 중요한 시스템 소프트웨어는 주로 C, C++, C\#과 같은 언어로 개발됩니다.

그러나 최근에는 하드웨어 성능의 향상과 효율적인 라이브러리 및 런타임 환경의 발전으로 인해, 일반적인 애플리케이션 수준에서는 언어 간 속도 차이가 크게 체감되지 않는 경우도 많습니다.
이로 인해 Python, Ruby, JavaScript와 같이 개발 생산성이 높은 언어들이 널리 사용되고 있습니다.

특히 머신러닝이나 딥러닝과 같이 고정된 수치 연산이 반복적으로 수행되는 분야에서는, 전체 프로그램의 제어 흐름은 Python으로 작성하고 성능이 중요한 연산 부분만을 C 또는 C++로 구현하여 결합하는 방식이 널리 활용되고 있습니다.
이러한 구조는 Python의 생산성과 저수준 언어의 성능을 동시에 활용할 수 있다는 장점을 가집니다.

\section{개발을 위해 필요한 준비물}

프로그래밍을 학습하고 실제로 코드를 작성하기 위해서는 적절한 개발 환경이 필요합니다.
기본적으로는 코드를 작성할 수 있는 텍스트 에디터와, 작성한 코드를 실행하고 결과를 확인할 수 있는 실행 환경이 필요합니다.

Python의 경우 인터프리터(interpreter)를 통해 한 줄씩 코드를 실행하며 결과를 확인할 수 있습니다.
이 방식은 간단한 문법 실험이나 짧은 코드의 동작을 확인하는 데에는 유용하지만, 여러 파일로 구성된 프로그램이나 규모가 큰 코드를 작성하기에는 적합하지 않습니다.

따라서 본 교재에서는 통합 개발 환경(IDE)의 역할을 수행할 수 있는 텍스트 에디터를 사용하여 Python 프로그램을 작성합니다.
그중에서도 \textbf{Visual Studio Code}는 가볍고 확장성이 뛰어나며, Python 개발을 위한 다양한 기능을 제공하기 때문에 초보자와 숙련자 모두에게 널리 사용되고 있습니다.

Visual Studio Code를 사용하면 코드 작성, 실행, 디버깅을 하나의 환경에서 수행할 수 있으며, 확장 기능(extension)을 통해 Python 인터프리터 관리, 코드 자동 완성, 문법 오류 표시 등 개발에 필요한 여러 기능을 손쉽게 사용할 수 있습니다.

\begin{notebox}
인터프리터(interpreter)는 프로그래밍 언어로 작성된 코드를 \emph{한 줄씩 읽어 해석하고 즉시 실행하는 프로그램}을 의미합니다.
Python은 대표적인 인터프리터 언어로, 코드를 작성한 후 별도의 컴파일 과정 없이 곧바로 실행 결과를 확인할 수 있습니다.

예를 들어 Python 인터프리터 환경에서는 다음과 같이 한 줄의 코드를 입력하고 즉시 실행 결과를 확인할 수 있습니다.
    \begin{lstlisting}
    >>> 1 + 2
    3
    \end{lstlisting}
\end{notebox}

\section{컴파일 언어와 인터프리터 언어}

프로그래밍 언어는 일반적으로 코드를 실행하는 방식에 따라 \emph{컴파일 언어}와 \emph{인터프리터 언어}로 구분할 수 있습니다.

컴파일 언어는 사람이 작성한 소스 코드를 실행 이전에 기계어(machine code)로 변환하는 컴파일(compile) 과정을 거칩니다.
이 과정에서 생성된 실행 파일은 운영체제에 의해 직접 실행되며, 실행 시점에는 별도의 해석 과정이 필요하지 않습니다.
대표적인 컴파일 언어로는 C, C++, Rust 등이 있습니다.

반면 인터프리터 언어는 소스 코드를 미리 기계어로 변환하지 않고, 실행 시점에 인터프리터가 코드를 한 줄씩 해석하며 실행합니다.
Python, JavaScript, Ruby 등이 대표적인 인터프리터 언어에 해당합니다.
이 방식은 코드 수정 후 즉시 실행할 수 있어 개발과 실험이 빠르다는 장점이 있습니다.

일반적으로 컴파일 언어는 한 번 컴파일된 이후에는 기계어를 직접 실행하기 때문에 실행 속도가 빠르고, 시스템 자원을 효율적으로 사용할 수 있습니다.
반면 인터프리터 언어는 실행 중에 코드를 해석해야 하므로 상대적으로 실행 속도가 느린 편입니다.

그러나 인터프리터 언어는 플랫폼에 독립적인 실행이 가능하고, 코드 수정과 테스트를 반복하는 개발 과정에서 높은 생산성을 제공한다는 장점이 있습니다. 
이러한 특성으로 인해 인터프리터 언어는 교육용 언어, 스크립트 언어, 그리고 데이터 분석 및 인공지능 분야에서 널리 활용되고 있습니다.

\section{심화: Python은 정말 인터프리터 언어일까?}
앞에서 살펴본 분류에 따르면 Python은 일반적으로 인터프리터 언어로 소개됩니다.
그러나 내부 동작 과정을 살펴보면, Python은 단순한 의미의 인터프리터 언어라기보다는 \emph{컴파일과 인터프리트 방식을 모두 사용하는 혼합 구조}를 가지고 있습니다.

Python 프로그램이 실행되면, 소스 코드(\texttt{.py} 파일)는 곧바로 기계어로 해석되는 것이 아니라 먼저 \emph{바이트코드(bytecode)}라는 중간 표현(intermediate representation)으로 변환됩니다.
이 과정은 Python 인터프리터 내부에서 자동으로 수행되며, 일반적으로 사용자가 직접 인식할 필요는 없습니다.

이렇게 생성된 바이트코드는 Python 가상 머신(Python Virtual Machine, PVM)에 의해 한 명령씩 해석되어 실행됩니다.
즉, Python은 실행 이전에 소스 코드를 바이트코드로 변환하는 일종의 컴파일 단계를 거친 뒤, 해당 바이트코드를 인터프리트하여 실행하는 구조를 가집니다.

이러한 구조 덕분에 Python은 컴파일 언어가 갖는 일부 장점과 인터프리터 언어의 편의성을 동시에 제공합니다.
예를 들어, 바이트코드를 재사용함으로써 반복 실행 시 성능을 일정 부분 향상시킬 수 있으며, 동시에 코드 수정 후 즉시 실행할 수 있는인터프리터 언어의 장점도 유지할 수 있습니다.

정리하면, Python은 전통적인 의미의 컴파일 언어도, 완전한 인터프리터 언어도 아닌, \emph{바이트코드 기반 인터프리터 언어}로 이해하는 것이 가장 적절합니다.

\section{다양한 Python 용어들}

프로그래밍을 학습하는 과정에서 용어를 정확히 이해하는 것은 매우 중요합니다.
본 절에서는 Python 프로그래밍에서 자주 등장하는 기본적인 용어들을 정리합니다.

\subsection{문장 (Statement)}

\textbf{문장(statement)}이란
\textbf{실행할 수 있는 코드의 최소 단위}를 의미합니다.
Python에서는 일반적으로 \emph{한 줄의 코드가 하나의 문장}이라고 생각해도 무방합니다.
여러 개의 문장이 모여 하나의 프로그램을 구성합니다.

\subsection{표현식 (Expression)}

\textbf{표현식(expression)}은
\textbf{어떠한 값을 만들어 내는 코드}를 의미합니다.
이때 값이란 숫자, 문자열, 계산 결과 등과 같은 구체적인 데이터를 의미합니다.

예를 들어 다음과 같은 코드는 모두 표현식입니다.

\begin{examplebox}
\begin{lstlisting}
1 + 2
"Hello"
3 * 4
\end{lstlisting}
\end{examplebox}


단, \texttt{+}, \texttt{-}와 같은 연산자는 그 자체만으로는 어떠한 값도 만들어내지 못하므로 표현식이 아닙니다.

\subsection{키워드 (Keyword)}

\textbf{키워드(keyword)}는
Python에서 \textbf{특별한 의미가 부여된 예약어}입니다.
키워드는 Python 언어 자체에서 이미 사용하기로 정해진 단어이므로, \textbf{변수명이나 함수명으로 사용할 수 없습니다}.

Python은 대소문자를 구분하는 언어입니다. 
예를 들어 \texttt{True}는 키워드이지만, \texttt{true}는 키워드가 아닙니다.

현재 사용 가능한 키워드 목록은 다음과 같이 확인할 수 있습니다.

\begin{examplebox}
\begin{lstlisting}
import keyword
print(keyword.kwlist)
\end{lstlisting}
\end{examplebox}

키워드는 암기할 필요는 없으며, 프로그래밍을 학습하다 보면 자연스럽게 익히게 됩니다.

\subsection{식별자 (Identifier)}

\textbf{식별자(identifier)}는
프로그래밍 언어에서 \textbf{이름을 붙일 때 사용하는 단어}로, 주로 변수, 함수, 클래스의 이름으로 사용됩니다.
식별자는 다음과 같은 규칙을 따라야 합니다.

\begin{itemize}
    \item 키워드를 사용할 수 없습니다.
    \item 특수 문자는 언더바(\_)만 사용할 수 있습니다.
    \item 숫자로 시작할 수 없습니다.
    \item 공백을 포함할 수 없습니다.
\end{itemize}

일반적으로 식별자는 영어로 작성하며, \texttt{a}, \texttt{b}와 같은 의미 없는 이름보다는 
\texttt{file}, \texttt{output}과 같이 의미를 잘 드러내는 이름을 사용하는 것이 좋습니다.

\subsubsection{스네이크 케이스와 캐멀 케이스}

식별자에는 공백을 사용할 수 없기 때문에, 여러 단어로 이루어진 이름을 표현할 때 다음과 같은 표기법을 사용합니다.

\begin{itemize}
    \item \textbf{스네이크 케이스 (snake\_case)}  
    단어 사이를 언더바(\_)로 연결하는 방식입니다.
    
    \item \textbf{캐멀 케이스 (CamelCase)}  
    각 단어의 첫 글자를 대문자로 표기하는 방식입니다.
\end{itemize}

\begin{examplebox}
\begin{lstlisting}
# snake_case
item_list = []
login_status = True

# CamelCase
ItemList = []
LoginStatus = True
\end{lstlisting}
\end{examplebox}

Python에서는 일반적으로 첫 글자가 소문자인 경우 스네이크 케이스, 첫 글자가 대문자인 경우 캐멀 케이스를 사용합니다.
캐멀 케이스는 주로 클래스 이름에 사용되며, 스네이크 케이스는 변수와 함수 이름에 사용됩니다.

\subsection{주석 (Comment)}

\textbf{주석(comment)}은
프로그램의 실행에는 영향을 주지 않으며, 코드를 설명하기 위해 사용됩니다.
Python에서는 \texttt{\#} 기호를 사용하여 한 줄 주석을 작성할 수 있습니다.

\begin{examplebox}
\begin{lstlisting}
# Simple Example
print("Hello, Python!")
\end{lstlisting}
\end{examplebox}

주석은 프로그램 실행 전에 제거되므로, 실행 결과에는 영향을 주지 않습니다.

\subsection{연산자와 자료 (Operator and Literal)}

\textbf{연산자(operator)}는
값과 값 사이에 특정한 연산을 적용하기 위해 사용하는 기호입니다.
연산자 자체는 값이 되지 않습니다.

\begin{examplebox}
\begin{lstlisting}
a = 1 + 1
b = 10 - 10
c = a + b
print(a, b, c, sep="/")
\end{lstlisting}
\end{examplebox}

\textbf{자료(data)} 또는 \textbf{리터럴(literal)}은 숫자, 문자열 등과 같이 \textbf{값 그 자체}를 의미합니다.

\section{Hello World!}

이제 여러분은 Python의 세계로 발을 디딜 준비가 되었습니다.
앞에서 Python의 특징과 개발 환경을 살펴보았으니, 이제 실제로 첫 번째 Python 프로그램을 실행해 보겠습니다.

프로그래밍 언어를 처음 배울 때, 가장 전통적으로 작성하는 프로그램은 \textbf{Hello World!}를 출력하는 것입니다.
이 간단한 예제를 통해 코드를 작성하고 실행하는 기본적인 흐름을 익힐 수 있습니다.

\begin{examplebox}
\begin{lstlisting}
print("Hello, World!")
\end{lstlisting}
\end{examplebox}

위 코드는 문자열 \texttt{"Hello, World!"}를 화면에 출력합니다.
여기서 \texttt{print}는 Python에서 값을 출력하기 위해 사용하는 기본적인 함수입니다.
괄호 안에 출력하고 싶은 내용을 작성하면, 해당 내용이 그대로 화면에 나타납니다.

\begin{notebox}
Python에서는 들여쓰기(indentation)가 단순한 가독성 요소가 아니라, \textbf{코드의 구조를 결정하는 문법 요소}입니다.
같은 수준으로 들여쓴 코드들은 하나의 블록(block)에 속하며, 들여쓰기가 잘못되면 문법 오류가 발생할 수 있습니다.
\end{notebox}

\begin{exercisebox}
자신의 이름과 학번을 출력하는 Python 프로그램을 작성해 보세요.
\end{exercisebox}

