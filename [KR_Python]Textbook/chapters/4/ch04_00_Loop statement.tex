\chapter{반복문(Loop statement)}
프로그래밍에서 특정 명령을 반복하기 위해서 같은 코드를 작성하는 것은 매우 비효율적입니다.
예를 들어 print() 함수를 반복해서 호출한다고 생각해봅시다.

\begin{examplebox}
\begin{lstlisting}
print("1")
print("2")
print("3")
\end{lstlisting}

출력 결과(예시):
\begin{verbatim}
1
2
3
\end{verbatim}
\end{examplebox}

위와 같이 유사한 형태의 코드를 여러 번 작성할 수도 있겠지만,
반복 횟수가 늘어날수록 코드는 길어지고, 유지보수도 어려워집니다.

예를 들어 출력해야 할 숫자가 3개가 아니라
100개, 혹은 1,000개라면
같은 \texttt{print()} 문을 그만큼 작성해야 할까요?
이는 현실적으로 불가능합니다.

이처럼 \textbf{같은 작업을 여러 번 반복해야 하는 상황}에서
프로그래밍 언어는 \textbf{반복문(loop statement)}이라는 문법을 제공합니다.

\begin{conceptbox}
반복문이란,
\textbf{특정 조건을 만족하는 동안}
\textbf{같은 코드 블록을 여러 번 실행하도록 하는 문법}입니다.
\end{conceptbox}

반복문을 사용하면
\begin{itemize}
    \item 코드의 길이를 획기적으로 줄일 수 있고
    \item 반복 횟수를 쉽게 변경할 수 있으며
    \item 수정과 확장이 훨씬 쉬운 프로그램을 작성할 수 있습니다.
\end{itemize}

Python에서는 대표적으로
\texttt{for} 반복문과 \texttt{while} 반복문을 제공합니다.
이번 장에서는 이 두 반복문의 기본 개념과 사용법을 차례대로 살펴봅니다.