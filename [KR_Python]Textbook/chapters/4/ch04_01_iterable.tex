% ---------------------------------------------------------
\section{이터러블(iterable)}

앞에서 반복문의 필요성에 대해 살펴보았습니다.
그렇다면 반복문은
\textbf{무엇을 기준으로 반복을 수행할까요?}

Python의 반복문, 특히 \texttt{for} 반복문은
아무 값이나 반복할 수 있는 것이 아니라,
\textbf{이터러블(iterable)}이라는 성질을 가진 대상에 대해서만
반복을 수행할 수 있습니다.

즉, \texttt{for} 문을 제대로 이해하기 위해서는
먼저 \textbf{이터러블이 무엇인지}를 이해해야 합니다.
이번 절에서는
\texttt{for} 문을 배우기에 앞서,
이터러블의 개념과 특징을 먼저 살펴봅니다.

\begin{conceptbox}
이터러블(iterable)이란,
\textbf{여러 개의 값을 순서대로 가지고 있으며}
\textbf{한 번에 하나씩 꺼낼 수 있는 객체}를 의미합니다.
이러한 객체는 \texttt{for} 문을 통해
앞에서부터 차례대로 반복할 수 있습니다.
\end{conceptbox}

% ---------------------------------------------------------
\subsection{이터러블의 핵심 개념}

이터러블의 개념은 다음 한 문장으로 요약할 수 있습니다.

\begin{conceptbox}
이터러블은
\textbf{``\texttt{for} 문으로 반복할 수 있는 대상''}입니다.
\end{conceptbox}

Python에서 \texttt{for} 반복문은
이터러블 안에 들어 있는 요소를
하나씩 변수에 담아가며
같은 코드 블록을 반복 실행합니다.

\begin{examplebox}
\begin{lstlisting}
for x in [1, 2, 3]:
    print(x)
\end{lstlisting}

출력 결과(예시):
\begin{verbatim}
1
2
3
\end{verbatim}
\end{examplebox}

위 예시에서
리스트 \texttt{[1, 2, 3]}은
여러 개의 값을 순서대로 가지고 있으므로
대표적인 이터러블입니다.

% ---------------------------------------------------------
\subsection{대표적인 이터러블 자료형}

Python에서 자주 사용되는 이터러블 자료형은 다음과 같습니다.

\begin{itemize}
    \item \texttt{list} : 여러 개의 값을 순서대로 저장
    \item \texttt{tuple} : 변경할 수 없는 리스트
    \item \texttt{str} : 문자들의 나열
    \item \texttt{dict} : key들의 집합
    \item \texttt{range} : 일정 범위의 정수 시퀀스
\end{itemize}

이 자료형들은 모두
\texttt{for} 문과 함께 사용할 수 있습니다.

\begin{examplebox}
\begin{lstlisting}
for ch in "Python":
    print(ch)
\end{lstlisting}

출력 결과(예시):
\begin{verbatim}
P
y
t
h
o
n
\end{verbatim}
\end{examplebox}

문자열은 여러 개의 문자로 이루어져 있기 때문에
문자 하나하나를 순서대로 꺼낼 수 있으며,
따라서 이터러블로 취급됩니다.

% ---------------------------------------------------------
\subsection{이터러블이 아닌 값}

반대로,
다음과 같은 값들은
이터러블이 아닙니다.

\begin{itemize}
    \item 정수(\texttt{int})
    \item 실수(\texttt{float})
    \item 불 값(\texttt{bool})
\end{itemize}

이 값들은
여러 개의 요소를 담고 있지 않기 때문에
\texttt{for} 문으로 반복할 수 없습니다.

\begin{examplebox}
\begin{lstlisting}
for x in 10:
    print(x)
\end{lstlisting}
\end{examplebox}

\begin{notebox}
이터러블이 아니라는 것은
``반복해서 하나씩 꺼낼 대상이 없다''는 의미에 가깝습니다.
따라서 \texttt{for} 문을 적용할 수 없습니다.
\end{notebox}

% ---------------------------------------------------------
\subsection{\texttt{range()}와 이터러블}

\texttt{range()} 함수는
반복문에서 가장 자주 사용되는
대표적인 이터러블입니다.

\begin{examplebox}
\begin{lstlisting}
for i in range(5):
    print(i)
\end{lstlisting}

출력 결과(예시):
\begin{verbatim}
0
1
2
3
4
\end{verbatim}
\end{examplebox}

\texttt{range(5)}는
숫자 0부터 4까지를
차례대로 생성하는 이터러블이며,
반복 횟수가 명확한 경우에 매우 유용합니다.

\begin{notebox}
\texttt{range()}는
모든 숫자를 미리 저장하는 것이 아니라,
필요할 때 하나씩 생성하는 방식으로 동작합니다.
따라서 메모리 사용 측면에서도 효율적입니다.
\end{notebox}

% ---------------------------------------------------------
\subsection{심화: 엄격한 의미에서의 이터러블}

지금까지는
이터러블을
``\texttt{for} 문으로 반복할 수 있는 대상''
이라는 관점에서 설명했습니다.
이 수준의 이해만으로도
대부분의 반복문을 사용하는 데에는 전혀 문제가 없습니다.

하지만 Python 내부 관점에서
이터러블은 보다 \textbf{엄격한 정의}를 가지고 있습니다.

\begin{conceptbox}
엄격한 의미에서
\textbf{이터러블(iterable)}이란,
\textbf{반복자(iterator)를 만들어낼 수 있는 객체}를 의미합니다.
\end{conceptbox}

즉,
이터러블 자체가
``하나씩 값을 꺼내는 역할''을 직접 수행하는 것이 아니라,
\textbf{반복을 담당하는 객체(iterator)를 제공할 수 있는 대상}입니다.

Python에서 \texttt{for} 문이 실행될 때의 내부 흐름을
개념적으로 표현하면 다음과 같습니다.

\begin{itemize}
    \item \texttt{for} 문은 먼저 이터러블 객체를 받는다
    \item 이터러블로부터 반복자(iterator)를 얻는다
    \item 반복자를 통해 값을 하나씩 꺼내며 반복을 수행한다
\end{itemize}

이러한 구조 덕분에
리스트, 문자열, \texttt{range}와 같이
겉보기에는 서로 전혀 다른 자료형들도
같은 \texttt{for} 문으로 반복할 수 있습니다.

\begin{notebox}
현재 단계에서는
``이터러블은 \texttt{for} 문으로 반복 가능한 대상''
이라고 이해해도 충분합니다.
반복자(iterator)와 내부 동작 방식은
이후 고급 문법이나 내부 구현을 다룰 때
자연스럽게 다시 등장하게 됩니다.
\end{notebox}

정리하면,
\begin{itemize}
    \item 이터러블은 반복 가능한 대상이다.
    \item 엄격하게는, 반복자를 만들어낼 수 있는 객체이다.
    \item \texttt{for} 문은 이터러블의 내부 구조를 몰라도 사용할 수 있도록 설계되어 있다.
\end{itemize}

\begin{readernotebox}
다음 질문에 답하며 이터러블 개념을 정리해 보세요.
\begin{itemize}
    \item 이터러블이란 무엇인가?
    \vspace{1.0cm}
    \item 문자열이 이터러블인 이유는 무엇인가?
    \vspace{1.0cm}
    \item \texttt{range(3)}은 어떤 값들을 차례대로 만들어내는가?
\end{itemize}
\end{readernotebox}
