% ---------------------------------------------------------
\section{while 반복문}

\texttt{while} 문은
\textbf{반복 횟수가 미리 정해져 있지 않은 경우}에 주로 사용됩니다.
대신,
\textbf{언제 반복을 멈출지에 대한 조건이 명확할 때}
적합한 반복문입니다.

즉, \texttt{for} 문이
``정해진 대상을 순회하는 반복문''이라면,
\texttt{while} 문은
``조건이 만족되는 동안 계속 반복하는 반복문''이라고 이해할 수 있습니다.

\begin{conceptbox}
\texttt{while} 문은
\textbf{조건식이 참(True)인 동안}
코드 블록을 반복 실행합니다.
조건식이 거짓(False)이 되는 순간,
반복은 즉시 종료됩니다.
\end{conceptbox}

% ---------------------------------------------------------
\subsection{\texttt{while} 문의 기본 형태}

\texttt{while} 문의 기본 구조는 다음과 같습니다.

\begin{examplebox}
\begin{lstlisting}
while condition_statement:
    # code block
    ...
\end{lstlisting}
\end{examplebox}

각 요소의 의미는 다음과 같습니다.

\begin{itemize}
    \item \texttt{condition\_statement} : 반복을 계속할지 판단하는 조건식
    \item 들여쓰기 된 코드 블록 : 조건이 참일 동안 반복 실행되는 부분
\end{itemize}

\begin{notebox}
\texttt{while} 문 역시
\textbf{콜론(\texttt{:})과 들여쓰기}가 문법의 일부입니다.
들여쓰기가 잘못되면
\texttt{IndentationError}가 발생합니다.
\end{notebox}

% ---------------------------------------------------------
\subsection{카운트를 이용한 \texttt{while} 문}

가장 기본적인 \texttt{while} 문의 사용 예시는
\textbf{카운트 변수(counter variable)}를 사용하는 방식입니다.

\begin{examplebox}
\begin{lstlisting}
count = 0

while count < 5:
    print(count)
    count += 1
\end{lstlisting}

출력 결과(예시):
\begin{verbatim}
0
1
2
3
4
\end{verbatim}
\end{examplebox}

\begin{notebox}
위 코드에서
\texttt{count += 1}이 없다면
\texttt{count < 5} 조건은 영원히 참이 되어
무한 반복에 빠지게 됩니다.
\end{notebox}

% ---------------------------------------------------------
\subsection{무한 반복(Infinite Loop)}

\texttt{while} 문은
조건을 항상 참으로 두면
\textbf{무한 반복(infinite loop)}을 만들 수 있습니다.

\begin{examplebox}
\begin{lstlisting}
while True:
    print("Running...")
\end{lstlisting}
\end{examplebox}

\begin{conceptbox}
\texttt{while True} 형태는
\textbf{의도적으로 무한 반복을 만들 때} 자주 사용됩니다.
이 경우,
반복을 종료하기 위해서는
반드시 \texttt{break} 문이 필요합니다.
\end{conceptbox}

\subsubsection{\texttt{break}를 이용한 종료}

\begin{examplebox}
\begin{lstlisting}
while True:
    x = int(input("Enter a number (0 to quit): "))
    if x == 0:
        break
    print(x)
\end{lstlisting}
\end{examplebox}

\begin{notebox}
이와 같은 구조는
사용자 입력, 메뉴 선택, 게임 루프 등에서
매우 자주 등장합니다.
\end{notebox}

% ---------------------------------------------------------
\subsection{\texttt{continue}와 \texttt{while} 문}

\texttt{continue}는
현재 반복을 즉시 종료하고
다음 반복으로 넘어가게 합니다.

\begin{examplebox}
\begin{lstlisting}
num = 0

while num < 10:
    num += 1
    if num % 2 == 0:
        continue
    print(num)
\end{lstlisting}

출력 결과(예시):
\begin{verbatim}
1
3
5
7
9
\end{verbatim}
\end{examplebox}

\begin{notebox}
\texttt{continue} 이후의 코드는 실행되지 않으므로,
조건과 변수 갱신의 순서에 특히 주의해야 합니다.
\end{notebox}

% ---------------------------------------------------------
\subsection{\texttt{for} 문과 \texttt{while} 문 비교}

\texttt{for} 문과 \texttt{while} 문은 서로 대체 가능한 경우도 있지만,
\textbf{의도에 따라 구분해서 사용하는 것이 중요}합니다.
\begin{itemize}
    \item \texttt{for} 문:
    반복 대상이나 횟수가 명확할 때
    \item \texttt{while} 문:
    종료 조건이 명확하고,
    반복 횟수가 실행 중에 결정될 때
\end{itemize}

\begin{notebox}
일반적으로
\textbf{가능하면 \texttt{for} 문을 먼저 고려하고},
\texttt{for} 문으로 표현하기 어려운 경우에
\texttt{while} 문을 사용하는 것이
가독성과 안전성 측면에서 좋습니다.
\end{notebox}

% ---------------------------------------------------------
\begin{exercisebox}
다음 문제를 풀어보세요.
\begin{enumerate}
    \item \texttt{while} 문을 사용하여
    1부터 10까지의 합을 구하세요.
    \item 사용자로부터 정수를 계속 입력받다가
    음수가 입력되면 반복을 종료하는 코드를 작성하세요.
    \item \texttt{while True}와 \texttt{break}를 사용하여
    간단한 메뉴 선택 구조를 만들어보세요.
\end{enumerate}
\end{exercisebox}

\begin{readernotebox}
이번 절에서 배운 내용을 정리해보세요.
\begin{itemize}
    \item \texttt{while} 문은 언제 사용하는가?
    \item 무한 반복은 언제 필요하며, 어떻게 종료하는가?
    \item \texttt{break}와 \texttt{continue}의 역할은 무엇인가?
    \item \texttt{for} 문과 \texttt{while} 문의 차이는 무엇인가?
\end{itemize}
\end{readernotebox}
