% =========================================================
% \section{튜플(Tuple)}
% =========================================================
\section{튜플(Tuple)}

여러 개의 값을 하나의 변수로 묶어
함께 다루고 싶을 때,
리스트(list) 외에도
\textbf{튜플(tuple)}을 사용할 수 있습니다.
튜플은 리스트와 매우 유사하지만,
\textbf{한 번 생성하면 내부 값을 바꿀 수 없는 자료형}이라는 점에서
중요한 차이가 있습니다.

\begin{conceptbox}
튜플(tuple)은 여러 개의 값을
하나의 변수에 담아 사용하는 자료형입니다.
튜플은 소괄호(\texttt{()})를 사용하여 생성하며,
각 값은 쉼표로 구분합니다.

튜플은 \textbf{불변(immutable)} 자료형이므로
생성 이후에 요소를 추가하거나 삭제하거나 변경할 수 없습니다.
\end{conceptbox}

\begin{notebox}
튜플 역시 리스트처럼 \textbf{순서(order)}가 있는 자료형입니다.
따라서 인덱스를 사용하여 요소에 접근할 수 있으며,
슬라이싱(slicing) 또한 지원합니다.
\end{notebox}

튜플의 기본적인 형태와 사용 예시는 다음과 같습니다.

\begin{examplebox}
\begin{lstlisting}
# define tuple
tuple_a = (123, 45, 6, "String", True)

print(tuple_a[0])
print(tuple_a[1:3])
\end{lstlisting}

출력 결과:
\begin{verbatim}
123
(45, 6)
\end{verbatim}
\end{examplebox}

\subsection{튜플 생성하기}

\subsubsection{괄호 생략과 튜플 패킹(packing)}
파이썬에서는 소괄호를 생략해도
쉼표(\texttt{,})가 있으면 튜플로 인식합니다.
이를 \textbf{튜플 패킹(tuple packing)}이라고 합니다.

\begin{examplebox}
\begin{lstlisting}
t = 1, 2, 3
print(t)
print(type(t))
\end{lstlisting}

출력 결과:
\begin{verbatim}
(1, 2, 3)
<class 'tuple'>
\end{verbatim}
\end{examplebox}

\subsubsection{원소가 하나인 튜플 (중요)}
원소가 하나인 튜플을 만들 때는
\textbf{반드시 쉼표가 필요}합니다.

\begin{examplebox}
\begin{lstlisting}
a = (5)
b = (5,)

print(a, type(a))
print(b, type(b))
\end{lstlisting}

출력 결과:
\begin{verbatim}
5 <class 'int'>
(5,) <class 'tuple'>
\end{verbatim}
\end{examplebox}

\begin{notebox}
\texttt{(5)}는 단순히 괄호로 감싼 정수이며,
\texttt{(5,)}만이 ``원소가 하나인 튜플''입니다.
\end{notebox}

\subsection{튜플 활용하기}

\subsubsection{인덱싱과 슬라이싱}
튜플은 시퀀스이므로 리스트와 동일하게
인덱싱/슬라이싱이 가능합니다.

\begin{examplebox}
\begin{lstlisting}
t = (10, 20, 30, 40, 50)

print(t[0])
print(t[-1])
print(t[1:4])
print(t[:3])
\end{lstlisting}

출력 결과:
\begin{verbatim}
10
50
(20, 30, 40)
(10, 20, 30)
\end{verbatim}
\end{examplebox}

\subsubsection{튜플 언패킹(unpacking)}
튜플은 여러 값을 한 번에 할당하는 데 매우 자주 쓰입니다.
이를 \textbf{튜플 언패킹(tuple unpacking)}이라 합니다.

\begin{examplebox}
\begin{lstlisting}
pair = (3, 5)
x, y = pair
print(x, y)
\end{lstlisting}

출력 결과:
\begin{verbatim}
3 5
\end{verbatim}
\end{examplebox}

튜플 언패킹은 값 교환(swap)에서도 매우 유용합니다.

\begin{examplebox}
\begin{lstlisting}
a, b = 10, 20
a, b = b, a
print(a, b)
\end{lstlisting}

출력 결과:
\begin{verbatim}
20 10
\end{verbatim}
\end{examplebox}

\subsubsection{확장 언패킹: \texttt{*} 사용하기}
\texttt{*}를 사용하면
여러 개의 값을 한꺼번에 받아올 수 있습니다.

\begin{examplebox}
\begin{lstlisting}
t = (1, 2, 3, 4, 5)
a, *mid, b = t

print(a)
print(mid)
print(b)
\end{lstlisting}

출력 결과:
\begin{verbatim}
1
[2, 3, 4]
5
\end{verbatim}
\end{examplebox}

\begin{notebox}
\texttt{*mid}로 모아진 값들은 \textbf{리스트(list)}로 저장됩니다.
\end{notebox}

\subsubsection{불변(immutable)한 튜플 자료형}
튜플은 불변 자료형이므로
인덱스로 요소를 직접 변경할 수 없습니다.

\begin{examplebox}
\begin{lstlisting}
t = (1, 2, 3)
t[0] = 10
\end{lstlisting}

출력 결과:
\begin{verbatim}
TypeError: 'tuple' object does not support item assignment
\end{verbatim}
\end{examplebox}

\begin{notebox}
튜플은 ``안전하게 변하지 않는 데이터''를 표현할 때 유용합니다.
또한 좌표 (x, y) 같은 \textbf{쌍(pair)}을 표현할 때 자주 사용됩니다.
\end{notebox}

\subsection{튜플 자료형 메서드}
튜플은 불변 자료형이므로
리스트처럼 많은 메서드를 제공하지 않습니다.
대표적으로 다음 두 메서드만 자주 사용합니다.

\subsubsection{\texttt{count(x)}}
\begin{conceptbox}
\texttt{count(x)}는 튜플에서 값이 \texttt{x}인 요소의 개수를 반환합니다.
\end{conceptbox}

\begin{examplebox}
\begin{lstlisting}
t = (1, 2, 2, 3)
print(t.count(2))
\end{lstlisting}

출력 결과:
\begin{verbatim}
2
\end{verbatim}
\end{examplebox}

\subsubsection{\texttt{index(x)}}
\begin{conceptbox}
\texttt{index(x)}는 튜플에서 값이 \texttt{x}인 요소의 첫 번째 인덱스를 반환합니다.
\end{conceptbox}

\begin{examplebox}
\begin{lstlisting}
t = ("a", "b", "c", "b")
print(t.index("b"))
\end{lstlisting}

출력 결과:
\begin{verbatim}
1
\end{verbatim}
\end{examplebox}

\begin{exercisebox}
다음 코드의 실행 결과를 예측해보세요.
\begin{enumerate}
    \item \begin{verbatim}
t = (1, 2, 3, 4, 5)
print(t[1:4])
print(t[-3:])
    \end{verbatim}
    \vspace{1.0cm}
    \item \begin{verbatim}
a = (5)
b = (5,)
print(type(a))
print(type(b))
    \end{verbatim}
    \vspace{1.0cm}
    \item 튜플이 리스트보다 유리한 상황을 2가지 이상 서술해보세요.
\end{enumerate}
\end{exercisebox}

\begin{readernotebox}
튜플의 핵심 개념(불변성, 언패킹, 원소 하나 튜플)을 정리해보세요.
\end{readernotebox}