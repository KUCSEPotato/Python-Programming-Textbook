% =========================================================
% \section{집합(Set)}
% =========================================================
\section{집합(Set)}

리스트와 튜플은
\textbf{중복을 허용}합니다.
하지만 어떤 상황에서는
중복이 없는 값들만 모아서 관리하거나,
값이 존재하는지 빠르게 확인하고 싶을 때가 있습니다.
이때 사용하는 자료형이 \textbf{집합(set)}입니다.

\begin{conceptbox}
집합(set)은 \textbf{중복을 허용하지 않는} 자료형입니다.
또한 집합은 \textbf{순서(order)가 없습니다}.
따라서 인덱스로 접근할 수 없으며,
대신 특정 값의 존재 여부를 빠르게 검사할 수 있습니다.

집합은 중괄호(\texttt{\{\}}) 또는 \texttt{set()}을 사용하여 생성합니다.
\end{conceptbox}

\begin{notebox}
빈 중괄호 \texttt{\{\}}는 집합이 아니라 \textbf{딕셔너리}입니다.
따라서 빈 집합을 만들려면 반드시 \texttt{set()}을 사용해야 합니다.
\end{notebox}

\subsection{집합 생성하기}

\subsubsection{기본 생성}
\begin{examplebox}
\begin{lstlisting}
s = {1, 2, 3, 3, 2}
print(s)
\end{lstlisting}

출력 결과:
\begin{verbatim}
{1, 2, 3}
\end{verbatim}
\end{examplebox}

\subsubsection{\texttt{set()} 생성자}
\begin{examplebox}
\begin{lstlisting}
a = set([1, 1, 2, 3])
b = set("banana")
print(a)
print(b)
\end{lstlisting}

출력 결과(예시):
\begin{verbatim}
{1, 2, 3}
{'b', 'a', 'n'}
\end{verbatim}
\end{examplebox}

\begin{notebox}
\texttt{set("banana")}는 문자열 전체를 원소로 넣는 것이 아니라,
문자 하나하나를 원소로 넣습니다.
문자열 전체를 원소로 넣고 싶다면 \texttt{\{"banana"\}}처럼 작성해야 합니다.
\end{notebox}

\subsection{집합 활용하기}

\subsubsection{원소 추가/삭제}
\begin{conceptbox}
집합은 \textbf{뮤터블(mutable)} 자료형입니다.
따라서 원소를 추가하거나 삭제할 수 있습니다.
\end{conceptbox}

\begin{examplebox}
\begin{lstlisting}
s = set()
s.add(1)
s.add(2)
s.add(2)
print(s)

s.remove(1)
print(s)
\end{lstlisting}

출력 결과:
\begin{verbatim}
{1, 2}
{2}
\end{verbatim}
\end{examplebox}

\begin{notebox}
\texttt{remove(x)}는 원소가 없으면 \texttt{KeyError}가 발생합니다.
원소가 없어도 에러 없이 넘어가고 싶다면 \texttt{discard(x)}를 사용합니다.
\end{notebox}

\subsubsection{존재 여부 확인 (멤버십 테스트)}
집합은 특정 값이 존재하는지 확인하는 연산이 매우 빠릅니다.

\begin{examplebox}
\begin{lstlisting}
s = {1, 2, 3}
print(2 in s)
print(5 in s)
\end{lstlisting}

출력 결과:
\begin{verbatim}
True
False
\end{verbatim}
\end{examplebox}

\subsection{집합 연산(합집합/교집합/차집합)}
집합은 수학의 집합처럼
합집합, 교집합, 차집합 등의 연산을 지원합니다.

\begin{conceptbox}
\begin{itemize}
    \item 합집합(union): \texttt{A | B} 또는 \texttt{A.union(B)}
    \item 교집합(intersection): \texttt{A \& B} 또는 \texttt{A.intersection(B)}
    \item 차집합(difference): \texttt{A - B} 또는 \texttt{A.difference(B)}
    \item 대칭차집합(symmetric difference): \texttt{A \^ B} 또는 \texttt{A.symmetric\_difference(B)}
\end{itemize}
\end{conceptbox}

\begin{examplebox}
\begin{lstlisting}
A = {1, 2, 3}
B = {3, 4, 5}

print(A | B)
print(A & B)
print(A - B)
print(A ^ B)
\end{lstlisting}

출력 결과(예시):
\begin{verbatim}
{1, 2, 3, 4, 5}
{3}
{1, 2}
{1, 2, 4, 5}
\end{verbatim}
\end{examplebox}

\subsection{집합 자료형 메서드}

\subsubsection{\texttt{add(x)}, \texttt{update(iterable)}}
\begin{conceptbox}
\texttt{add(x)}는 원소 하나를 추가하고,
\texttt{update(iterable)}는 여러 원소를 한 번에 추가합니다.
\end{conceptbox}

\subsubsection{\texttt{remove(x)}, \texttt{discard(x)}, \texttt{pop()}}
\begin{conceptbox}
\texttt{remove(x)}는 원소를 삭제하며, 없으면 에러가 발생합니다.
\texttt{discard(x)}는 원소가 없어도 에러가 발생하지 않습니다.
\texttt{pop()}은 임의의 원소 하나를 제거하여 반환합니다.
\end{conceptbox}

\subsubsection{부분집합/상위집합}
\begin{conceptbox}
집합은 포함 관계를 검사할 수 있습니다.
\begin{itemize}
    \item \texttt{A <= B}: $A$가 $B$의 부분집합
    \item \texttt{A < B}: $A$가 $B$의 진부분집합
\end{itemize}
\end{conceptbox}

\subsection{집합 원소의 조건: 해시 가능(hashable)}
\begin{conceptbox}
집합의 원소도 딕셔너리의 키와 마찬가지로
\textbf{해시 가능(hashable)}해야 합니다.
일반적으로 \textbf{불변(immutable)} 자료형만 원소가 될 수 있습니다.
\end{conceptbox}

\begin{examplebox}
\begin{lstlisting}
s = set()
s.add((1, 2))   # tuple OK
print(s)
\end{lstlisting}
\end{examplebox}

\begin{examplebox}
\begin{lstlisting}
s = set()
s.add([1, 2])   # list not allowed
\end{lstlisting}

출력 결과:
\begin{verbatim}
TypeError: unhashable type: 'list'
\end{verbatim}
\end{examplebox}

\subsection{실전에서 집합을 쓰는 대표 상황}
\begin{itemize}
    \item \textbf{중복 제거}: \texttt{set(list)}로 중복 제거
    \item \textbf{빠른 존재 검사}: 방문 체크(visited), 포함 여부 검사
    \item \textbf{두 데이터의 공통/차이 비교}: 교집합/차집합 연산 활용
\end{itemize}

\begin{exercisebox}
\begin{enumerate}
    \item 빈 집합을 \texttt{\{\}}로 만들 수 없는 이유를 설명해보세요.
    \vspace{1.0cm}
    \item 다음 코드의 실행 결과를 예측해보세요.
\begin{verbatim}
A = {1, 2, 3, 4}
B = {3, 4, 5}
print(A & B)
print(A - B)
print(A | B)
\end{verbatim}
    \vspace{1.0cm}
    \item \texttt{remove}와 \texttt{discard}의 차이를 설명해보세요.
\end{enumerate}
\end{exercisebox}

\begin{readernotebox}
집합의 핵심 개념(중복 없음, 순서 없음, 멤버십 검사, 집합 연산)을 정리해보세요.
\end{readernotebox}