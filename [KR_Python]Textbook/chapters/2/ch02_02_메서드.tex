% ========================================
% 2.2
% ========================================
\section{메서드(Method)}

각 자료형에는 해당 자료형을 다루기 쉽도록 미리 정의된 전용 함수들이 존재합니다.
이러한 함수들을 \textbf{메서드(Method)}라고 부릅니다.

Python에서 메서드란, 특정 객체(object)에 속하여
그 객체의 데이터를 직접 다룰 수 있는 함수입니다.
좀 더 쉽게 말하면, 메서드는 \textit{객체가 수행할 수 있는 동작}이라고 이해할 수 있습니다.

앞으로 살펴볼 다양한 자료형들은 각각 고유한 메서드들을 가지고 있으며,
이를 통해 해당 자료형의 특성에 맞는 연산을
보다 직관적이고 일관된 방식으로 수행할 수 있습니다.

메서드는 특히 \textbf{객체가 가지고 있는 데이터(상태)를 자연스럽게 처리하는 데}에 적합합니다.
따라서 이후 자료형을 학습할 때에는,
각 자료형이 어떤 메서드를 제공하는지에 주목하며 살펴보는 것이 중요합니다.

\begin{notebox}
    메서드는 동작 방식에 따라,
    \textbf{객체의 상태를 직접 변경하는 메서드(in-place)}와
    \textbf{새로운 값을 반환하는 메서드}로 나눌 수 있습니다.
\end{notebox}

\begin{readernotebox}
아래 공간에 지금까지 배운 내용들을 자유롭게 정리해 보세요.
(필자는 자료형에 대한 구분을 트리형태로 정리해보길 권합니다.)

\end{readernotebox}
    
\newpage