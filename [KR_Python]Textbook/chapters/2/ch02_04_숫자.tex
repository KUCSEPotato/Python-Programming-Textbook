% ========================================
% 2.4
% ========================================
\section{숫자 자료형 (Number)}

Python에서 수는 단순한 계산 대상이 아니라,
프로그램에서 다양한 연산과 처리를 수행하기 위한
\textbf{기본적인 자료형} 중 하나입니다.
그러나 수학에서 배운 것처럼,
숫자는 하나의 종류로만 이루어져 있지 않으며
여러 기준에 따라 구분할 수 있습니다.

그중에서도 가장 익숙한 구분은
\textbf{정수(integer)}와 \textbf{실수(floating-point number)}입니다.
Python 역시 이러한 수학적 개념을 반영하여
정수형과 실수형을 서로 다른 자료형으로 제공합니다.

그렇다면 이 둘은 어떤 기준으로 구분될까요?
Python에서는 숫자의 \textbf{표현 방식}, 특히
\textbf{소수점의 존재 여부}에 따라
정수형과 실수형을 구분합니다.

소수점이 없는 숫자는 정수형으로,
소수점이 포함된 숫자는 실수형으로 처리됩니다.
이때 \texttt{0}은 정수형이고,
\texttt{0.0}은 실수형이라는 점에 주의해야 합니다.
두 값은 수학적으로는 같아 보일 수 있지만,
Python에서는 서로 다른 자료형으로 취급됩니다.

% ========================================
% 2.4.1
% ========================================
\subsection{정수형 (integer)}

\begin{conceptbox}
정수형(\texttt{int})은 소수점이 없는 수를 표현하는 자료형입니다.
Python의 정수형은 양의 정수, 음의 정수, 그리고 0을 모두 포함하며,
자릿수의 크기에 제한이 없는 것이 특징입니다.
\end{conceptbox}

수학에서 사용하는 정수 개념과 마찬가지로,
정수형은 주로 개수, 순서, 횟수와 같이
\textbf{정확한 값이 필요한 상황}에서 사용됩니다.

% ========================================
% 2.4.1.1
% ========================================
\subsubsection{정수형의 기본 예}

\begin{examplebox}
\begin{lstlisting}
a = 10
b = -3
c = 0

print(a)
print(b)
print(c)
\end{lstlisting}

출력 결과:
\begin{verbatim}
10
-3
0
\end{verbatim}
\end{examplebox}

% ========================================
% 2.4.1.2
% ========================================
\subsubsection{정수형의 연산}

\begin{conceptbox}
정수형은 사칙연산과 같은 기본적인 수치 연산을 지원합니다.
정수형끼리의 연산 결과는
연산의 종류에 따라 정수 또는 실수가 될 수 있습니다.
\end{conceptbox}

\begin{table}[h]
\centering
\renewcommand{\arraystretch}{1.3}
\begin{tabular}{l l p{6.5cm}}
\hline
\textbf{연산자} & \textbf{의미} & \textbf{설명} \\
\hline
\texttt{+} & 덧셈 & 두 정수를 더한 결과를 반환합니다. \\

\texttt{-} & 뺄셈 & 두 정수의 차를 계산합니다. \\

\texttt{*} & 곱셈 & 두 정수를 곱한 결과를 반환합니다. \\

\texttt{/} & 나눗셈 & 나눗셈을 수행하며, 결과는 항상 실수형(\texttt{float})입니다. \\

\texttt{//} & 정수 나눗셈 & 나눗셈의 몫만 계산하며, 결과는 정수형입니다. \\

\texttt{\%} & 나머지 & 나눗셈의 나머지를 계산합니다. \\

\texttt{**} & 거듭제곱 & 왼쪽 값을 오른쪽 값만큼 거듭제곱합니다. \\
\hline
\end{tabular}
\caption{정수형(\texttt{int})에서 사용되는 주요 산술 연산자}
\end{table}

\begin{examplebox}
\begin{lstlisting}
x = 7
y = 2

print(x + y)
print(x - y)
print(x * y)
print(x // y)
print(x % y)
\end{lstlisting}

출력 결과:
\begin{verbatim}
9
5
14
3
1
\end{verbatim}
\end{examplebox}

\begin{notebox}
정수 나눗셈에서 \texttt{//} 연산자는
몫만을 계산하는 연산자이며,
결과는 항상 정수형입니다.
\end{notebox}

% ========================================
% 2.4.2
% ========================================
\subsection{실수형 (floating-point number)}

\begin{conceptbox}
실수형(\texttt{float})은 소수점을 포함하는 수를 표현하는 자료형입니다.
Python의 실수형은 수학적 실수를 컴퓨터 내부에서 \textbf{근사값}으로 표현합니다.
또한 실수형에서도 정수형과 동일한 산술 연산자(\texttt{+}, \texttt{-}, \texttt{*}, \texttt{/} 등)를 사용할 수 있으며,
사용 방법 또한 동일합니다.
\end{conceptbox}

실수형은 측정값, 비율, 평균과 같이
\textbf{소수점 단위의 값이 필요한 상황}에서 자주 사용됩니다.

% ========================================
% 2.4.2.1
% ========================================
\subsubsection{실수형의 기본 예}

\begin{examplebox}
\begin{lstlisting}
a = 3.14
b = 0.5
c = -2.0

print(a)
print(b)
print(c)
\end{lstlisting}

출력 결과:
\begin{verbatim}
3.14
0.5
-2.0
\end{verbatim}
\end{examplebox}

위 예시와 같이 소수점을 사용하여 실수형을 표현할 수 있으며,
또한 \textbf{지수 표기법(scientific notation)}을 사용하여
실수형을 표현할 수도 있습니다.
이때 \texttt{e}는 10의 거듭제곱을 의미합니다.

\begin{examplebox}
\begin{lstlisting}
x = 1.23e2
y = 4.5e-1

print(x)
print(y)
\end{lstlisting}

출력 결과:
\begin{verbatim}
123.0
0.45
\end{verbatim}
\end{examplebox}

Python에서 실수형을 지수 표기법(\texttt{e})으로 표현할 수 있도록 한 이유는
매우 크거나 매우 작은 수를
\textbf{효율적이고 명확하게 표현하기 위해서}입니다.

과학 계산, 공학 계산, 머신러닝과 같은 분야에서는
$10^{10}$, $10^{-9}$와 같이
크기의 차이가 큰 수들이 자주 등장합니다.
이러한 수를 일반적인 소수점 형태로 표현하면
자릿수가 길어져 가독성이 떨어지고,
의미를 한눈에 파악하기 어렵습니다.

지수 표기법을 사용하면
수의 크기와 비율을 간결하게 표현할 수 있으며,
실수형의 내부 표현 방식과도 잘 부합합니다.
실제로 실수형은 메모리에서
지수와 가수 형태로 저장되므로,
지수 표기법은 이러한 구조를
사람이 이해하기 쉽게 표현한 방법이라고 볼 수 있습니다.
\begin{notebox}
메모리와 관련하여 더 자세한 내용은 2.4.3을 참고해주세요.
\end{notebox}

% ========================================
% 2.4.2.2
% ========================================
\subsubsection{정수형과 실수형의 연산}

\begin{conceptbox}
정수형과 실수형이 함께 연산될 경우,
Python은 자동으로 정수형을 실수형으로 변환하여
연산을 수행합니다.
이를 \textbf{형 변환(type conversion)}이라고 합니다.
\end{conceptbox}

\begin{examplebox}
\begin{lstlisting}
x = 5
y = 2.0

print(x + y)
print(x * y)
\end{lstlisting}

출력 결과:
\begin{verbatim}
7.0
10.0
\end{verbatim}
\end{examplebox}

\begin{notebox}
서로 다른 자료형의 값을
정의되지 않은 방식으로 연산하려 하면
\texttt{TypeError} 예외가 발생합니다.

예를 들어,
문자열과 숫자 자료형을
\texttt{+} 연산자로 연산하려 할 경우,
문자열은 \texttt{+}를 문자열 연결 연산자로 해석하고,
숫자는 덧셈 연산자로 해석하려 하기 때문에
연산의 의미가 충돌하게 됩니다.

\begin{lstlisting}
str1 = "Python"
int1 = 123
print(str1 + int1)
\end{lstlisting}

출력 결과:
\begin{verbatim}
TypeError: can only concatenate str (not "int") to str
\end{verbatim}
\end{notebox}

위 노트와 형 변환은 구분되는 것임을 기억해주세요.

% ========================================
% 2.4.2.3
% ========================================
\subsubsection{실수형의 주의사항}

\begin{conceptbox}
실수형은 컴퓨터 내부에서 이진수 기반의 근사값으로 표현되기 때문에,
일부 실수 연산에서는
기대와 다른 결과가 나타날 수 있습니다.
\end{conceptbox}

\begin{examplebox}
\begin{lstlisting}
a = 0.1
b = 0.2

print(a + b)
\end{lstlisting}

출력 결과:
\begin{verbatim}
0.30000000000000004
\end{verbatim}
\end{examplebox}

\begin{notebox}
이와 같은 현상은 실수형의 표현 방식에 따른 것으로,
프로그램의 오류가 아닙니다.
머신러닝이나 수치 계산에서는
이러한 특성을 고려하여 실수를 다루는 것이 중요합니다.
\end{notebox}

% ========================================
% 2.4.3
% ========================================
\subsection{심화: 숫자 자료형이 메모리에 저장되는 방식}

컴퓨터의 메모리는 모든 데이터를 비트(bit) 단위로 저장합니다.
숫자 자료형 역시 예외가 아니며,
정수형과 실수형 모두 메모리에서는
0과 1로 이루어진 비트들의 나열로 표현됩니다.
다만, 같은 숫자라 하더라도
어떤 자료형으로 해석하느냐에 따라
메모리에 저장되는 구조와 해석 방식은 달라집니다.
즉, 숫자 자료형은 값의 종류뿐만 아니라
\textbf{메모리에 저장되고 해석되는 방식까지 포함하는 개념}이라고 볼 수 있습니다.

특히 실수형의 경우,
모든 실수를 무한히 정확하게 표현할 수 없기 때문에
\textbf{한정된 개수의 비트}를 사용하여 값을 저장합니다.
이로 인해 실수형은 실제 값을 그대로 저장하는 것이 아니라,
가장 가까운 값으로 근사하여 표현하게 됩니다.
이러한 제한된 비트 표현 방식 때문에
일부 실수 연산에서는
기대와 다른 결과가 나타날 수 있으며,
이는 컴퓨터가 실수를 저장하는 구조적 특성에서 비롯된 현상입니다.

\begin{notebox}
해당 내용에 대해 더 깊이 알고 싶은 독자는
추가적인 자료를 참고하기 바랍니다.
본 교재에서는 개념적 이해에 필요한 범위까지만 다루며,
세부적인 내용은 다루지 않습니다.
\end{notebox}

\begin{readernotebox}
다음 질문에 답하며 정수형과 실수형을 정리해 보세요.

\begin{itemize}
    \item 정수형과 실수형의 차이점은 무엇인가?
    \vspace{1.2cm}

    \item 정수형끼리의 나눗셈에서 \texttt{//} 연산자를 사용하는 이유는 무엇인가?
    \vspace{1.2cm}

    \item 실수형 연산에서 발생할 수 있는 오차는 왜 생기는가?
    \vfill
\end{itemize}
\end{readernotebox}

% ========================================
% 2.4.4
% ========================================
\subsection{숫자 자료형 메서드}

지금까지 정수형과 실수형의 개념과 연산 방법을 살펴보았습니다.
문자열과 마찬가지로 숫자 자료형 역시
값을 처리하기 위한 여러 기능을 제공합니다.
다만 숫자 자료형의 경우,
문자열처럼 다양한 전용 메서드를 가지기보다는
\textbf{내장 함수}의 형태로 기능이 제공되는 경우가 많습니다.

이번 절에서는 숫자 자료형과 함께 자주 사용되는
대표적인 함수와 메서드들을 살펴봅니다.

\begin{notebox}
이 절에서는 편의상
\texttt{abs()}, \texttt{round()}와 같은 기능을
숫자 자료형의 \textbf{메서드}로 소개하고 있습니다.
이는 숫자 값을 처리하는 기능이라는 점에서
문자열 메서드와 역할이 유사하기 때문입니다.

다만 엄밀히 말하면,
이러한 기능들은
\texttt{숫자.메서드()} 형태가 아니라
\texttt{함수(숫자)} 형태로 호출되는
\textbf{내장 함수}입니다.
즉, 호출 방식은 함수이지만,
개념적으로는 숫자 객체를 처리하는 메서드 역할을 한다고
이해하면 됩니다.
\end{notebox}

\subsubsection{절댓값 함수 \texttt{abs()}}

\begin{conceptbox}
\texttt{abs()} 함수는 숫자의 절댓값을 반환합니다.
정수형과 실수형 모두에 대해 사용할 수 있으며,
부호를 제거한 크기만을 얻고자 할 때 사용합니다.
\end{conceptbox}

\begin{examplebox}
\begin{lstlisting}
print(abs(-5))
print(abs(3.2))
\end{lstlisting}

출력 결과:
\begin{verbatim}
5
3.2
\end{verbatim}
\end{examplebox}

\subsubsection{반올림 함수 \texttt{round()}}

\begin{conceptbox}
\texttt{round()} 함수는 실수 값을 반올림한 결과를 반환합니다.
두 번째 인자를 사용하면
소수점 이하 자릿수를 지정할 수 있습니다.
\end{conceptbox}

\begin{examplebox}
\begin{lstlisting}
print(round(3.14159))
print(round(3.14159, 2))
\end{lstlisting}

출력 결과:
\begin{verbatim}
3
3.14
\end{verbatim}
\end{examplebox}

\begin{notebox}
실수형은 근사값으로 저장되기 때문에,
\texttt{round()}의 결과가
직관과 다르게 나타나는 경우도 있을 수 있습니다.
\end{notebox}

그렇다면 Python은 왜 은행가 반올림을 채택하고 있을까요?

\begin{notebox}
은행가 반올림은 통계 계산이나
대규모 수치 연산에서
반올림 오차가 누적되는 것을 줄이기 위해 사용됩니다.
이 때문에 Python은 \texttt{round()} 함수에
이 방식을 채택하고 있습니다.
\end{notebox}

\subsubsection{심화: 일반적인 반올림과 \texttt{round()} 함수의 차이}

\begin{conceptbox}
일반적으로 수학에서 사용하는 반올림은
소수점 아래 첫 자리가 5 이상이면 올리고,
4 이하이면 내리는 방식(사사오입)을 의미합니다.
그러나 Python의 \texttt{round()} 함수는
이와는 다른 규칙을 사용합니다.
\end{conceptbox}

Python의 \texttt{round()} 함수는
\textbf{은행가 반올림(bankers' rounding)} 또는
\textbf{짝수 반올림(round half to even)} 방식을 따릅니다.
이 방식에서는 반올림하려는 값이 정확히 0.5인 경우,
가장 가까운 \textbf{짝수}로 반올림합니다.

\begin{examplebox}
\begin{lstlisting}
print(round(2.5))
print(round(3.5))
print(round(4.5))
print(round(5.5))
\end{lstlisting}

출력 결과:
\begin{verbatim}
2
4
4
6
\end{verbatim}
\end{examplebox}

위 결과에서 볼 수 있듯이,
\texttt{round(2.5)}는 3이 아니라 2로,
\texttt{round(3.5)}는 4로 반올림됩니다.
이는 반올림 결과가 특정 방향으로 치우치는 것을 방지하기 위한 설계입니다.

\begin{notebox}
일반적인 사사오입 반올림은 다음으로 배울 메서드를 이용하여 구현할 수 있습니다.
\end{notebox}

\subsubsection{정수형 변환 메서드 \texttt{int()}}

\begin{conceptbox}
\texttt{int()} 함수는 숫자나 문자열을
정수형으로 변환합니다.
실수형을 정수형으로 변환할 경우,
소수점 이하 값은 버려집니다.
\end{conceptbox}

\begin{examplebox}
\begin{lstlisting}
print(int(3.7))
print(int(-2.9))
\end{lstlisting}

출력 결과:
\begin{verbatim}
3
-2
\end{verbatim}
\end{examplebox}

\begin{notebox}
\texttt{int()} 함수는 반올림을 수행하지 않으며,
단순히 소수점 이하를 제거합니다.
\end{notebox}

\begin{exercisebox}
입력이 \texttt{x}라고 할 때,
\texttt{int()} 함수를 사용하여
사사오입 반올림을 구현해 보세요.
정답은 다음 노트박스에 제시되어 있으니,
정답을 보기 전에
먼저 스스로 해결해 보시기 바랍니다.
\end{exercisebox}

\newpage

\begin{notebox}
앞의 문제를 스스로 해결해 보셨나요?
양의 실수에 대해서는,
반올림하고자 하는 값에 \texttt{0.5}를 더한 뒤
정수형으로 변환하는 방식으로
사사오입 반올림을 구현할 수 있습니다.

다음은 그 예시입니다.

\begin{verbatim}
int(x + 0.5)
\end{verbatim}

이 식이 왜 사사오입 반올림처럼 동작하는지,
직접 수치를 대입하여
계산 과정을 따라가며 이해해 보시기 바랍니다.

단, 이 방식은 음수에 대해서는
일반적인 반올림과 다르게 동작할 수 있으므로
주의가 필요합니다.
\end{notebox}

\begin{readernotebox}
자유롭게 지금까지의 내용을 정리해보세요.

\end{readernotebox}

\subsubsection{실수형 변환 메서드 \texttt{float()}}

\begin{conceptbox}
\texttt{float()} 함수는 정수형이나 문자열을
실수형으로 변환합니다.
계산 결과를 실수형으로 처리하고자 할 때
자주 사용됩니다.
\end{conceptbox}

\begin{examplebox}
\begin{lstlisting}
print(float(5))
print(float("3.14"))
\end{lstlisting}

출력 결과:
\begin{verbatim}
5.0
3.14
\end{verbatim}
\end{examplebox}

\subsubsection{정수형의 진법 관련 메서드}

\begin{conceptbox}
정수형은 서로 다른 진법으로 표현할 수 있으며,
Python은 이를 위한 함수들을 제공합니다.
\end{conceptbox}

\begin{itemize}
    \item \texttt{bin()} : 2진수 문자열로 변환
    \item \texttt{oct()} : 8진수 문자열로 변환
    \item \texttt{hex()} : 16진수 문자열로 변환
\end{itemize}

\begin{examplebox}
\begin{lstlisting}
print(bin(10))
print(oct(10))
print(hex(10))
\end{lstlisting}

출력 결과:
\begin{verbatim}
0b1010
0o12
0xa
\end{verbatim}
\end{examplebox}

\begin{notebox}
2진수, 8진수, 16진수는
컴퓨터 내부에서 숫자를 표현하거나
메모리 구조를 이해하는 데 자주 사용됩니다.
특히 2진수는
컴퓨터가 데이터를 저장하고 처리하는
기본적인 표현 방식입니다.

또한 Python에서 진법 변환 함수의 출력 결과에는
각 진법을 나타내는 접두어가 함께 붙습니다.
\texttt{0b}는 2진수,
\texttt{0o}는 8진수,
\texttt{0x}는 16진수를 의미합니다.
\end{notebox}


\newpage