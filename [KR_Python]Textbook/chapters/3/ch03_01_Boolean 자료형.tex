% ========================================
% Conditional Statement Chapter Materials
% ========================================

\section{불(Boolean) 자료형}

프로그래밍에서 ``조건''은 결국 \textbf{참(True)} 또는 \textbf{거짓(False)}으로 판단됩니다.
이를 표현하기 위한 기본 자료형이 \textbf{불(boolean) 자료형}입니다.
Boolean은 ``불리언'' 또는 ``불린''이라고 부르며, 교재에서는 보통 \textbf{불}이라고 표현하겠습니다.

\begin{conceptbox}
불(\texttt{bool}) 자료형은 \textbf{오직 두 값}만 가집니다.
\[
\texttt{True},\ \texttt{False}
\]
\end{conceptbox}

\begin{examplebox}
\begin{lstlisting}
print(True)
print(False)
print(type(True))
\end{lstlisting}

출력 결과(예시):
\begin{verbatim}
True
False
<class 'bool'>
\end{verbatim}
\end{examplebox}

불 값은 비교 연산, 논리 연산, 그리고 조건문(\texttt{if})에서 핵심적으로 사용됩니다.

% ---------------------------------------------------------
\subsection{비교 연산자(Comparison Operators)}

\begin{conceptbox}
\textbf{비교 연산자}는 두 값의 상등 여부(같다/다르다)나 대소 관계(크다/작다)를 비교하여
\textbf{불 값을 반환하는 연산자}입니다.
\end{conceptbox}

\begin{table}[H]
\centering
\begin{tabular}{c c \quad c c}
\hline
\textbf{연산자} & \textbf{의미} & \textbf{연산자} & \textbf{의미} \\
\hline
\texttt{==} & 같다 & \texttt{>} & 크다 \\
\texttt{!=} & 다르다 & \texttt{>=} & 크거나 같다 \\
\texttt{<} & 작다 & \texttt{<=} & 작거나 같다 \\
\hline
\end{tabular}
\caption{Python 비교 연산자}
\end{table}

\begin{notebox}
비교 연산자는 주로 \textbf{숫자}에 대해 사용하지만, \textbf{문자열}에도 적용할 수 있습니다.
문자열 비교는 ``사전식(lexicographic)'' 비교이며,
알파벳은 유니코드 순서에 따라 비교됩니다.
(한글 비교 역시 유니코드 순서의 영향을 받으므로, 일상적 ``사전 순서''와 완전히 같다고 단정하지는 않는 것이 안전합니다.)
\end{notebox}

\subsubsection{연속 비교(Chained Comparison)}

Python은 다음처럼 \textbf{비교식을 연속으로 연결}할 수 있습니다.

\begin{examplebox}
\begin{lstlisting}
x = 50
print(40 < x <= 50)
print(30 < x < 50)
\end{lstlisting}

출력 결과:
\begin{verbatim}
True
False
\end{verbatim}
\end{examplebox}

\begin{conceptbox}
\texttt{40 < x <= 50}는
\texttt{(40 < x) and (x <= 50)}와 같은 의미로 동작합니다.
\end{conceptbox}

% ---------------------------------------------------------
\subsection{논리 연산자(Logical Operators)}

\begin{conceptbox}
\textbf{논리 연산자}는 불 값들 사이의 논리 관계를 계산하여
\textbf{불 값을 반환}합니다.
\end{conceptbox}

\begin{table}[H]
\centering
\begin{tabular}{c c p{8cm}}
\hline
\textbf{연산자} & \textbf{의미} & \textbf{설명} \\
\hline
\texttt{not} & 아니다 & 불 값을 반대로 뒤집습니다. \\
\texttt{and} & 그리고 & 두 값이 모두 참일 때만 \texttt{True}입니다. \\
\texttt{or}  & 또는 & 둘 중 하나라도 참이면 \texttt{True}입니다. \\
\hline
\end{tabular}
\caption{Python 논리 연산자}
\end{table}

\subsubsection{단항/이항 연산자}

\begin{conceptbox}
\textbf{단항(unary) 연산자}는 피연산자가 1개,
\textbf{이항(binary) 연산자}는 피연산자가 2개인 연산자입니다.
\texttt{not}은 단항 연산자, \texttt{and/or}는 이항 연산자입니다.
\end{conceptbox}

\subsubsection{\texttt{not} 연산자}

\begin{examplebox}
\begin{lstlisting}
flag = True
print(not flag)

x = 10
print(not (x > 0))
\end{lstlisting}

출력 결과(예시):
\begin{verbatim}
False
False
\end{verbatim}
\end{examplebox}

\begin{notebox}
\textbf{중요:} Python에는 C 계열 언어처럼 \texttt{!} 연산자가 존재하지 않습니다.
부정을 나타낼 때는 반드시 \texttt{not}을 사용해야 합니다.
\end{notebox}

\subsubsection{\texttt{and} / \texttt{or} 연산자와 진리표}

\begin{conceptbox}
\texttt{and}는 \textbf{둘 다 참}이어야 참,
\texttt{or}는 \textbf{하나라도 참}이면 참입니다.
\end{conceptbox}

\begin{table}[H]
\centering
\begin{tabular}{c c c}
\hline
\textbf{A} & \textbf{B} & \textbf{A and B} \\
\hline
True & True & True \\
True & False & False \\
False & True & False \\
False & False & False \\
\hline
\end{tabular}
\caption{\texttt{and} 진리표}
\end{table}

\begin{table}[H]
\centering
\begin{tabular}{c c c}
\hline
\textbf{A} & \textbf{B} & \textbf{A or B} \\
\hline
True & True & True \\
True & False & True \\
False & True & True \\
False & False & False \\
\hline
\end{tabular}
\caption{\texttt{or} 진리표}
\end{table}

\begin{notebox}
Python의 \texttt{and/or}는 ``단락 평가(short-circuit)''를 수행합니다.
예를 들어 \texttt{A and B}에서 A가 거짓이면 B는 평가하지 않습니다.
이는 조건문에서 오류를 피하거나(예: \texttt{x != 0 and 10/x > 2})
불필요한 계산을 줄이는 데 유용합니다.
\end{notebox}

\begin{readernotebox}
이번 절에서 배운 Boolean 개념을 스스로 정리해보세요.

\begin{itemize}
    \item Boolean 자료형이란 무엇이며, 어떤 값만을 가지는가?
    \vspace{0.8cm}

    \item 비교 연산자는 무엇이며, 어떤 상황에서 Boolean 값을 반환하는가?
    \vspace{0.8cm}

    \item 논리 연산자 \texttt{and}, \texttt{or}, \texttt{not}의 동작 방식을 설명해보세요.
    \vspace{0.8cm}
    
\end{itemize}
\end{readernotebox}
