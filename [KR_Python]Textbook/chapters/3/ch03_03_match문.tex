% ---------------------------------------------------------
\section{match 문 (Pattern Matching)}

지금까지 우리는 조건 분기를 위해
\texttt{if}, \texttt{elif}, \texttt{else} 문을 사용해 왔습니다.
이 방식은 대부분의 상황에서 충분히 강력하지만,
조건의 경우의 수가 많아질수록
코드가 길어지고 구조를 한눈에 파악하기 어려워질 수 있습니다.

Python 3.10부터는 이러한 문제를 보완하기 위해
\textbf{\texttt{match} 문}이라는 새로운 조건 분기 문법이 도입되었습니다.
\texttt{match} 문은 하나의 값을 여러 \textbf{패턴(pattern)}과 비교하여,
처음으로 일치하는 경우의 코드만 실행합니다.

\begin{notebox}
\texttt{match} 문은 \textbf{Python 3.10 이상}에서만 사용할 수 있습니다.
사용 중인 Python 버전이 낮은 경우,
\texttt{match} 문은 문법 오류(SyntaxError)를 발생시킵니다.
\end{notebox}

% ---------------------------------------------------------
\subsection{match 문의 기본 개념}

\begin{conceptbox}
\texttt{match} 문은 하나의 값(\textit{subject})을 기준으로,
여러 \texttt{case} 패턴과 차례대로 비교합니다.
처음으로 매칭되는 \texttt{case} 블록만 실행되며,
이후의 \texttt{case}는 검사하지 않습니다.
\end{conceptbox}

겉으로 보기에는 C나 Java의 \texttt{switch} 문과 비슷해 보일 수 있지만,
Python의 \texttt{match} 문은 단순한 값 비교를 넘어서
\textbf{값의 구조(structure)를 검사하고},
그 일부를 \textbf{변수로 추출}할 수 있다는 점에서 차이가 있습니다.
이러한 방식은 다른 언어에서는
``패턴 매칭(pattern matching)''이라고 불립니다.

% ---------------------------------------------------------
\subsection{리터럴 값 매칭}

가장 단순한 형태의 \texttt{match} 문은
값을 여러 리터럴(literal)과 비교하는 방식입니다.

\begin{examplebox}
\begin{lstlisting}
status = 404

match status:
    case 400:
        print("Bad request")
    case 404:
        print("Not found")
    case 418:
        print("I'm a teapot")
    case _:
        print("Unknown error")
\end{lstlisting}

출력 결과(예시):
\begin{verbatim}
Not found
\end{verbatim}
\end{examplebox}

\begin{notebox}
\texttt{case \_}에서 사용된 \texttt{\_}는
\textbf{와일드카드(wildcard)}로,
앞의 어떤 패턴에도 매칭되지 않았을 때
항상 선택되는 기본 case 역할을 합니다.
따라서 \texttt{case \_}는 보통 가장 마지막에 위치합니다.
\end{notebox}

% ---------------------------------------------------------
\subsection{여러 값을 한 번에 처리하기 (\texttt{|})}

\texttt{match} 문에서는
여러 개의 값을 하나의 case에서 처리할 수 있습니다.
이를 위해 \texttt{|} 기호를 사용합니다.

\begin{conceptbox}
\texttt{|}는 논리 연산자의 ``or''와 비슷한 의미를 가지며,
여러 리터럴 중 하나라도 일치하면 해당 case가 실행됩니다.
\end{conceptbox}

\begin{examplebox}
\begin{lstlisting}
code = 403

match code:
    case 401 | 403 | 404:
        print("Not allowed")
    case _:
        print("Allowed")
\end{lstlisting}

출력 결과(예시):
\begin{verbatim}
Not allowed
\end{verbatim}
\end{examplebox}

이 방식은 \texttt{if} 문에서
\texttt{or} 연산자를 여러 번 사용하는 경우보다
코드를 더 간결하게 만들어 줍니다.

% ---------------------------------------------------------
\subsection{구조 매칭과 값 추출}

\texttt{match} 문의 가장 강력한 특징은
값의 \textbf{구조}를 검사하면서,
그 일부를 변수로 저장할 수 있다는 점입니다.

\begin{conceptbox}
패턴이 튜플이나 리스트의 형태를 가질 경우,
매칭이 성공하면 해당 위치의 값이
자동으로 변수에 저장됩니다.
이를 \textbf{바인딩(binding)}이라고 합니다.
\end{conceptbox}

\begin{examplebox}
\begin{lstlisting}
point = (3, 0)

match point:
    case (0, 0):
        print("Origin")
    case (0, y):
        print(f"Y={y}")
    case (x, 0):
        print(f"X={x}")
    case (x, y):
        print(f"X={x}, Y={y}")
    case _:
        print("Not a point")
\end{lstlisting}

출력 결과(예시):
\begin{verbatim}
X=3
\end{verbatim}
\end{examplebox}

위 예시에서 \texttt{(x, 0)} 패턴은
``두 번째 값이 0인 튜플''을 의미하며,
첫 번째 값은 변수 \texttt{x}에 저장됩니다.
이처럼 \texttt{match} 문은
값을 비교하는 동시에
필요한 정보를 자연스럽게 꺼내올 수 있습니다.

% ---------------------------------------------------------
\subsection{함수를 사용한 예제에 대한 설명}

\begin{notebox}
일부 자료나 문서에서는 \texttt{match} 문 예제를
\texttt{def}로 정의된 함수 형태로 제시하기도 합니다.
하지만 본 교재에서는 아직
\textbf{함수(function) 문법을 배우지 않았기 때문에},
함수의 구조나 동작 원리를 이해할 필요는 없습니다.

지금 단계에서는
``\texttt{match} 문이 조건 분기를 수행한다''는 점과
``패턴에 따라 다른 코드가 실행된다''는 개념만 이해하면 충분합니다.
함수에 대해서는 이후 장에서 자세히 다룰 예정입니다.
\end{notebox}

% ---------------------------------------------------------
\subsection{if 문과 match 문의 비교}

\begin{conceptbox}
\texttt{if} 문은 조건식을 직접 작성하여 분기하는 방식이고,
\texttt{match} 문은 하나의 값을 기준으로
여러 패턴을 비교하는 방식입니다.
\end{conceptbox}

조건이 단순하거나 논리식이 중요한 경우에는
\texttt{if} 문이 더 적합할 수 있으며,
값의 경우의 수가 많거나
구조를 기준으로 분기해야 하는 경우에는
\texttt{match} 문이 더 읽기 쉬운 코드를 만들어 줍니다.

\begin{notebox}
초보자 단계에서는
\texttt{if} 문을 충분히 익히는 것이 가장 중요합니다.
\texttt{match} 문은 이후 학습 단계에서
코드를 더 깔끔하게 정리하기 위한 도구로
천천히 익혀도 충분합니다.
\end{notebox}

\newpage
\begin{exercisebox}
다음 조건을 만족하도록 \texttt{match} 문을 작성해보세요.
    \begin{itemize}
        \item (0, 0) $\rightarrow$ \texttt{"Origin"}
        \item (0, y) $\rightarrow$ \texttt{"On Y-axis"}
        \item (x, 0) $\rightarrow$ \texttt{"On X-axis"}
        \item (x, y) $\rightarrow$ \texttt{"Point in plane"}
    \end{itemize}
    \vspace{1.2cm}
\end{exercisebox}

\begin{exercisebox}
다음 \texttt{match} 문의 실행 결과를 예측해보세요.
\begin{verbatim}
value = 3

match value:
    case 1:
        print("One")
    case 2 | 3:
        print("Two or Three")
    case _:
        print("Other")
\end{verbatim}
\end{exercisebox}

\begin{readernotebox}
\texttt{match} 문에 대해 스스로 정리해보세요.

\begin{itemize}
    \item \texttt{case \_}는 왜 필요한가?
    \vspace{1.0cm}

    \item 여러 값을 하나의 \texttt{case}에서 처리하려면 어떻게 해야 하는가?
    \vspace{1.0cm}

    \item 단순한 값 비교에서 \texttt{if} 문보다 \texttt{match} 문이 더 좋은 경우는 언제인가?
\end{itemize}
\end{readernotebox}