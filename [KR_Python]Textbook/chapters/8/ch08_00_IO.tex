% =========================================================
\chapter{입력과 출력 (Input \& Output)}

프로그램은
외부로부터 데이터를 입력받고,
처리한 결과를 다시 외부로 출력하는 과정을 반복합니다.
이러한 과정을 통틀어
\textbf{입력과 출력(Input / Output, I/O)}이라고 부릅니다.

출력은 단순히 화면에 값을 표시하는 것일 수도 있고,
파일에 데이터를 저장하거나,
다른 프로그램과 데이터를 주고받기 위한 준비 단계일 수도 있습니다.
이 장에서는
파이썬에서 데이터를 \textbf{표현하고, 출력하고, 저장하고, 다시 읽는 방법}을
체계적으로 정리합니다.

% ---------------------------------------------------------
\section{출력의 기본}

지금까지 가장 많이 사용한 출력 방법은
\texttt{print()} 함수입니다.

\begin{examplebox}
\begin{lstlisting}
x = 3
y = 5
print(x, y)
\end{lstlisting}

출력 결과(예시):
\begin{verbatim}
3 5
\end{verbatim}
\end{examplebox}

\texttt{print()}는
여러 값을 자동으로 문자열로 변환한 뒤
공백으로 구분하여 출력합니다.
간단한 디버깅이나 확인 용도로는 충분하지만,
출력 형식을 세밀하게 제어하기에는 한계가 있습니다.

% ---------------------------------------------------------
\section{출력 포매팅이 필요한 이유}

다음과 같은 경우를 생각해 봅시다.

\begin{itemize}
  \item 실수 값을 소수점 몇 자리까지만 출력하고 싶을 때
  \item 숫자들을 표 형태로 정렬하고 싶을 때
  \item 문자열과 변수 값을 자연스럽게 섞어 출력하고 싶을 때
\end{itemize}

이러한 요구를 만족시키기 위해
파이썬은 여러 가지 출력 포매팅 방식을 제공합니다.

% ---------------------------------------------------------
\section{포맷 문자열 리터럴 (f-문자열)}

가장 권장되는 출력 포매팅 방식은
\textbf{포맷 문자열 리터럴},
즉 \textbf{f-문자열}입니다.

\begin{conceptbox}
f-문자열은
문자열 앞에 \texttt{f} 또는 \texttt{F}를 붙이고,
\texttt{\{ \}} 안에 파이썬 표현식을 작성하여
해당 표현식의 값을 문자열에 삽입하는 방식입니다.
\end{conceptbox}

\begin{examplebox}
\begin{lstlisting}
year = 2016
event = "Referendum"
print(f"Results of the {year} {event}")
\end{lstlisting}

출력 결과(예시):
\begin{verbatim}
Results of the 2016 Referendum
\end{verbatim}
\end{examplebox}

\subsection{숫자 포매팅}

\begin{examplebox}
\begin{lstlisting}
import math
print(f"The value of pi is approximately {math.pi:.3f}.")
\end{lstlisting}

출력 결과(예시):
\begin{verbatim}
The value of pi is approximately 3.142.
\end{verbatim}
\end{examplebox}

\subsection{폭 지정과 정렬}

\begin{examplebox}
\begin{lstlisting}
table = {'Sjoerd': 4127, 'Jack': 4098, 'Dcab': 7678}
for name, phone in table.items():
    print(f"{name:10} ==> {phone:6d}")
\end{lstlisting}

출력 결과(예시):
\begin{verbatim}
Sjoerd     ==>   4127
Jack       ==>   4098
Dcab       ==>   7678
\end{verbatim}
\end{examplebox}

\subsection{변환 지정자}

\begin{examplebox}
\begin{lstlisting}
animals = "eels"
print(f"My hovercraft is full of {animals}.")
print(f"My hovercraft is full of {animals!r}.")
\end{lstlisting}

출력 결과(예시):
\begin{verbatim}
My hovercraft is full of eels.
My hovercraft is full of 'eels'.
\end{verbatim}
\end{examplebox}

\subsection{자기 설명식 표현식}

\begin{examplebox}
\begin{lstlisting}
bugs = "roaches"
count = 13
area = "living room"
print(f"Debugging {bugs=} {count=} {area=}")
\end{lstlisting}

출력 결과(예시):
\begin{verbatim}
Debugging bugs='roaches' count=13 area='living room'
\end{verbatim}
\end{examplebox}

% ---------------------------------------------------------
\section{\texttt{str()}과 \texttt{repr()}}

모든 객체는
문자열로 변환될 수 있습니다.
이때 주로 사용되는 함수가
\texttt{str()}과 \texttt{repr()}입니다.

\begin{itemize}
  \item \texttt{str()}: 사람이 읽기 좋은 표현
  \item \texttt{repr()}: 인터프리터가 이해할 수 있는 표현
\end{itemize}

\begin{examplebox}
\begin{lstlisting}
s = "Hello, world"
print(str(s))
print(repr(s))
\end{lstlisting}

출력 결과(예시):
\begin{verbatim}
Hello, world
'Hello, world'
\end{verbatim}
\end{examplebox}

\subsection{디버깅에서의 활용}

\begin{examplebox}
\begin{lstlisting}
x = 10 * 3.25
y = 200 * 200
print("x =", repr(x), ", y =", repr(y))
\end{lstlisting}

출력 결과(예시):
\begin{verbatim}
x = 32.5 , y = 40000
\end{verbatim}
\end{examplebox}

% ---------------------------------------------------------
\section{문자열 \texttt{format()} 메서드}

f-문자열 이전부터 사용되던 방식이
\texttt{str.format()} 메서드입니다.

\begin{examplebox}
\begin{lstlisting}
print("We are the {} who say {}!".format("knights", "Ni"))
\end{lstlisting}

출력 결과(예시):
\begin{verbatim}
We are the knights who say Ni!
\end{verbatim}
\end{examplebox}

\subsection{위치 인자와 키워드 인자}

\begin{examplebox}
\begin{lstlisting}
print("{0} and {1}".format("spam", "eggs"))
print("{food} is {adj}".format(food="spam", adj="horrible"))
\end{lstlisting}

출력 결과(예시):
\begin{verbatim}
spam and eggs
spam is horrible
\end{verbatim}
\end{examplebox}

% ---------------------------------------------------------
\section{수동 문자열 포매팅}

문자열 메서드를 이용해
출력을 직접 제어할 수도 있습니다.

\begin{examplebox}
\begin{lstlisting}
for x in range(1, 6):
    print(repr(x).rjust(2), repr(x*x).rjust(4))
\end{lstlisting}

출력 결과(예시):
\begin{verbatim}
 1    1
 2    4
 3    9
 4   16
 5   25
\end{verbatim}
\end{examplebox}

\begin{notebox}
이 방식은 가능하지만,
가독성과 유지보수 측면에서는
f-문자열이나 \texttt{format()} 방식이 더 권장됩니다.
\end{notebox}

% ---------------------------------------------------------
\section{옛 문자열 포매팅 (\texttt{\%})}

\begin{examplebox}
\begin{lstlisting}
import math
print("The value of pi is approximately %5.3f." % math.pi)
\end{lstlisting}

출력 결과(예시):
\begin{verbatim}
The value of pi is approximately 3.142.
\end{verbatim}
\end{examplebox}

\begin{notebox}
이 방식은
과거 코드에서 여전히 볼 수 있지만,
새로운 코드에서는 권장되지 않습니다.
\end{notebox}

% ---------------------------------------------------------
\section{파일 읽고 쓰기}

파일을 다루기 위해
\texttt{open(filename, mode, encoding)} 함수를 사용합니다.

\begin{examplebox}
\begin{lstlisting}
f = open("example.txt", "w", encoding="utf-8")
f.write("Hello, file!\n")
f.close()
\end{lstlisting}
\end{examplebox}

\subsection{\texttt{with} 문 사용}

\begin{examplebox}
\begin{lstlisting}
with open("example.txt", "r", encoding="utf-8") as f:
    print(f.read())
\end{lstlisting}

출력 결과(예시):
\begin{verbatim}
Hello, file!
\end{verbatim}
\end{examplebox}

% ---------------------------------------------------------
\section{파일 객체 메서드}

\subsection{\texttt{read()}, \texttt{readline()}, 반복}

\begin{examplebox}
\begin{lstlisting}
with open("example.txt", encoding="utf-8") as f:
    for line in f:
        print(line, end="")
\end{lstlisting}
\end{examplebox}

\subsection{\texttt{write()}, \texttt{seek()}, \texttt{tell()}}

\begin{examplebox}
\begin{lstlisting}
with open("binary.bin", "wb+") as f:
    f.write(b"0123456789")
    f.seek(5)
    print(f.read(1))
\end{lstlisting}

출력 결과(예시):
\begin{verbatim}
b'5'
\end{verbatim}
\end{examplebox}

% ---------------------------------------------------------
\section{JSON으로 구조적 데이터 저장}

\begin{conceptbox}
JSON은
리스트와 딕셔너리 같은
구조적인 데이터를 저장하고 교환하기 위한
표준 텍스트 형식입니다.
\end{conceptbox}

\subsection{저장}

\begin{examplebox}
\begin{lstlisting}
import json
data = {"name": "Alice", "age": 20}
with open("data.json", "w", encoding="utf-8") as f:
    json.dump(data, f)
\end{lstlisting}
\end{examplebox}

\subsection{불러오기}

\begin{examplebox}
\begin{lstlisting}
import json
with open("data.json", encoding="utf-8") as f:
    print(json.load(f))
\end{lstlisting}

출력 결과(예시):
\begin{verbatim}
{'name': 'Alice', 'age': 20}
\end{verbatim}
\end{examplebox}

% ---------------------------------------------------------
\begin{readernotebox}
입력과 출력 단원의 핵심을 스스로 점검해 보세요.
\begin{itemize}
  \item f-문자열이
        다른 출력 포매팅 방식보다
        권장되는 이유는 무엇인가요?
  \item \texttt{str()}과 \texttt{repr()}의 차이는
        어떤 상황에서 중요해지나요?
  \item 파일을 열 때
        \texttt{with} 문을 사용하는 이유는 무엇인가요?
  \item 텍스트 파일과
        바이너리 파일은
        언제 각각 사용해야 하나요?
  \item JSON이
        구조적 데이터를 저장하는 데
        적합한 이유는 무엇인가요?
\end{itemize}
\end{readernotebox}