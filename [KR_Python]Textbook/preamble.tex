% --------------------
% Encoding & Language
% --------------------
\usepackage[utf8]{inputenc}
\usepackage[T1]{fontenc}
\usepackage{kotex}
\usepackage{lmodern}

% --------------------
% Layout
% --------------------
\usepackage{geometry}
\geometry{margin=1in}

\usepackage{setspace}
\onehalfspacing

\usepackage{caption}
\setcounter{secnumdepth}{3}
\usepackage{forest}

% 문단 간격(교재용 추천)
\setlength{\parskip}{4pt}
\setlength{\parindent}{0pt}

% 페이지 하단 여백이 과도하게 늘어나는 현상 완화
\raggedbottom

% --------------------
% Math & Graphics
% --------------------
\usepackage{amsmath, amssymb, amsthm}
\usepackage{graphicx}
\usepackage{float}

% --------------------
% Hyperlink
% --------------------
\usepackage{hyperref}
\usepackage{cleveref}

% --------------------
% Lists
% --------------------
\usepackage{enumitem}
\setlist{nosep}

% --------------------
% Colors & Boxes
% --------------------
\usepackage{xcolor}
\usepackage[most]{tcolorbox}

% 박스 공통 간격(원하면 여기만 바꾸면 됨)
\newcommand{\boxskip}{8pt}

\newtcolorbox{conceptbox}{
    breakable,
    before skip=\boxskip,
    after skip=\boxskip,
    colback=blue!5,
    colframe=blue!60!black,
    title=Concept,
    fonttitle=\bfseries,
}

\newtcolorbox{examplebox}{
    breakable,
    before skip=\boxskip,
    after skip=\boxskip,
    colback=green!5,
    colframe=green!60!black,
    title=Example,
    fonttitle=\bfseries,
}

\newtcolorbox{notebox}{
    breakable,
    before skip=\boxskip,
    after skip=\boxskip,
    colback=yellow!10,
    colframe=yellow!60!black,
    title=Note,
    fonttitle=\bfseries,
}

\newtcolorbox{exercisebox}{
    breakable,
    before skip=\boxskip,
    after skip=\boxskip,
    colback=red!5,
    colframe=red!60!black,
    title=Exercise,
    fonttitle=\bfseries,
}

% "남은 여백을 채우기" 같은 강제 height는 공백 폭탄을 만들 수 있어 제거
% 대신 breakable + skip 통일로 레이아웃 안정화
\newtcolorbox{readernotebox}[1][]{
  enhanced,
  breakable=false,
  colback=white,
  colframe=black,
  boxrule=0.8pt,
  arc=3pt,
  left=10pt,
  right=10pt,
  top=8pt,
  bottom=8pt,
  title=필기 노트,
  before=\clearpage,
  after=\clearpage,
  height=\textheight,
  valign=top,
  #1
}




% --------------------
% Python Code Listings
% --------------------
\usepackage{listings}

\definecolor{codegray}{rgb}{0.95,0.95,0.95}
\definecolor{keywordblue}{rgb}{0.2,0.2,0.7}

\lstset{
    inputencoding=utf8,
    extendedchars=true
}

\lstdefinestyle{pythonstyle}{
    language=Python,
    backgroundcolor=\color{codegray},
    basicstyle=\ttfamily\small,
    keywordstyle=\color{keywordblue}\bfseries,
    commentstyle=\color{gray},
    stringstyle=\color{teal},
    numbers=left,
    numberstyle=\tiny\color{gray},
    numbersep=6pt,
    frame=single,
    breaklines=true,
    showstringspaces=false,
    columns=fullflexible,
    keepspaces=true,
}

\lstset{style=pythonstyle}
